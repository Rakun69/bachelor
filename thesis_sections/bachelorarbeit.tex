\documentclass[english,bibtotoc,liststotoc,oneside,BCOR=5mm,DIV=12]{scrbook}
\recalctypearea

\usepackage[english]{babel}
\usepackage[T1]{fontenc}
\usepackage{lmodern}
\usepackage{microtype}
\usepackage{amsmath}
\usepackage{amssymb}
\usepackage{amsthm}
\usepackage{graphicx}
\usepackage{float}
\usepackage{url}
\usepackage{hyperref}
\usepackage[nameinlink]{cleveref}
\usepackage{pgfplots}
\pgfplotsset{compat=1.18}
\usepackage{listings, color}
\usepackage{subcaption}
\usepackage[automark]{scrlayer-scrpage}
\setlength{\marginparwidth}{2cm}
\usepackage{todonotes}
\usepackage{comment}
\usepackage{tikz}
\usetikzlibrary{shapes,arrows,positioning,fit,backgrounds,shadows,decorations.pathmorphing,decorations.pathreplacing}
\usetikzlibrary{calc,intersections,through,backgrounds,matrix}
\usetikzlibrary{arrows.meta, calc, positioning, shapes.geometric}
\usepackage{algorithm}
\usepackage{algpseudocode}
\usepackage{tabularx}
\usepackage{array}
\newcolumntype{Y}{>{\raggedright\arraybackslash}X}
\usepackage[table]{xcolor}
\usepackage[most]{tcolorbox}
\tcbuselibrary{listings,skins}
% Neutral panel with teal accent for algorithms/notes
\newtcolorbox{algwithnotes}[1][]{enhanced,breakable,sharp corners,boxrule=0.6pt,
  colback=blue!3!white,colframe=black!20,borderline west={2pt}{0pt}{teal!70!black},
  title={#1},fonttitle=\bfseries}

% Listings style for nicer code blocks
\lstdefinestyle{codeblock}{%
  basicstyle=\ttfamily\small,
  breaklines=true,
  columns=fullflexible,
  keepspaces=true,
  showstringspaces=false,
  frame=single,
  framerule=0.5pt,
  backgroundcolor=\color{blue!5!white},
  rulecolor=\color{teal!70!black},
  numbers=left,
  numberstyle=\tiny,
  numbersep=16pt,
  xleftmargin=2.6em,
  framexleftmargin=2.0em
}
\lstset{style=codeblock}

% Theorem-like environments
\newtheorem{definition}{Definition}

\usepackage{booktabs}
\usepackage{csquotes}
\usepackage[backend=biber, style=ieee]{biblatex}
\addbibresource{bib/references.bib}

% Only number and list up to sections (no subsection/subsubsection numbering in ToC or headings)
\setcounter{secnumdepth}{1}
\setcounter{tocdepth}{1}

\clearpairofpagestyles
\chead[Verifiable Data Transformations in IoT Environments – Ramón Felipe Kühne]{Verifiable Data Transformations in IoT Environments – Ramón Felipe Kühne}
% Unified footer for all page styles (plain and scrheadings)
\cfoot[TU Berlin, 2025 \quad | \quad \thepage]{TU Berlin, 2025 \quad | \quad \thepage}
\KOMAoptions{headsepline=.4pt}
\renewcommand*{\chapterpagestyle}{scrheadings}
\pagestyle{scrheadings}

\graphicspath{{./img/}{../data/visualizations/}}

%%%%%%%%%%%%%%%%%%%%%%%%%%%%%%%%%%%%%%%%%%%%%%%%%%%%%%%%%%%%%%%%%%%%%%%%%%%%%%%%%%%%%%%%%%%%%%%%%%%%
%%%%%%%%%%%%%%%%%%%%%%%%%%%%%%%%%%%%%%%%%%%%%%%%%%%%%%%%%%%%%%%%%%%%%%%%%%%%%%%%%%%%%%%%%%%%%%%%%%%%
%%%%%%%%%%%%%%%%%%%%%%%%%%%%%%%%%%%%%%%%%%%%%%%%%%%%%%%%%%%%%%%%%%%%%%%%%%%%%%%%%%%%%%%%%%%%%%%%%%%%

\begin{document}

\frontmatter
\thispagestyle{empty}
\begin{center}

\vspace*{1.4cm}
{\LARGE \textbf{Technische Universität Berlin}}

\vspace{0.5cm}

{\large Information Systems Engineering\\[1mm]}

Fakultät IV\\
Einsteinufer 17\\
10587 Berlin\\
https://www.tu.berlin/ise\\

\vspace*{1cm}

\includegraphics[width=4cm]{tu_logo.png}

\vspace*{1.0cm}

{\LARGE Thesis}\\

\vspace{1.0cm}
{\LARGE \textbf{Verifiable Data Transformations in IoT}}\\
\vspace*{0.3cm}
{\LARGE \textbf{Environments using Recursive zk-SNARKs}}\\
\vspace*{1.0cm}
{\LARGE Ramón Felipe Kühne}
\\
\vspace*{0.5cm}
Matriculation Number: 456119\\
\vspace*{1.0cm}

\textbf{First Examiner:} Prof. Dr.-Ing. Stefan Tai\\
\textbf{Second Examiner:} Prof. Dr. Sahin Albayrak

\vspace*{0.3cm}

\textbf{Supervised by}\\
Karl Wolf\\
Fabian Piper

\end{center}


    \thispagestyle{empty}
    \cleardoublepage
    
\input{misc/self-assertion}
    \thispagestyle{empty}
    \cleardoublepage

% Switch to arabic numbering for all following pages
\mainmatter
\pagenumbering{arabic}
\setcounter{page}{1}

\section*{Declaration of Authorship}
I hereby declare that I have written this thesis independently and have not used any sources or aids other than those stated. All passages taken directly or indirectly from the published or unpublished work of others have been identified as such. This thesis has not been submitted in substantially the same form to any other examination board and has not been published.

\clearpage

\section*{Abstract}
\noindent

\clearpage

\section*{Acknowledgements}

\clearpage

\tableofcontents
\clearpage
\cleardoublepage
\listoffigures
\clearpage
\listoftables
\clearpage

\section*{List of Abbreviations}

\clearpage

%%%%%%%%%%%%%%%%%%%%%%%%%%%%%%%%%%%%%%%%%%%%%%%%%%%%%%%%%%%%%%%%%%%%%%%%%%%%%%%%%%%%%%%%%%%%%%%%%%%%
%%%%%%%%%%%%%%%%%%%%%%%%%%%%%%%%%%%%%%%%%%%%%%%%%%%%%%%%%%%%%%%%%%%%%%%%%%%%%%%%%%%%%%%%%%%%%%%%%%%%
%%%%%%%%%%%%%%%%%%%%%%%%%%%%%%%%%%%%%%%%%%%%%%%%%%%%%%%%%%%%%%%%%%%%%%%%%%%%%%%%%%%%%%%%%%%%%%%%%%%%

\chapter{Introduction}
\label{sec:intro}

\section{Motivation}
\label{sec:motivation}
IoT deployments such as smart home sensors, industrial monitors, and environmental sensing networks generate continuous high resolution time series data. To reduce communication and storage demands on resource constrained edge devices, this data is often aggregated in batches, for example via summation or averaging. Conventional aggregation lacks formal guarantees regarding data integrity and privacy \cite{SmartMeterPrivacySurvey2023, AStudyOnPrivacyPreserving2021}. Research has shown that even coarse patterns like hourly energy usage can expose private habits \cite{muellerGewinnungVerhaltensprofilenAm2010}. Aggregation pipelines that rely on unverified local computation are susceptible to tampering or omission, which undermines trust in reported values \cite{bohliPrivacyModelSmart2010aa}. This combination of emerging privacy threats and trust issues highlights the need for cryptographic verification mechanisms in IoT aggregation systems.

Global data production has exploded over the past decade. In 2010 approximately 2 zettabytes of data existed. By 2023 that number reached about 120 zettabytes. In 2024 it rose to approximately 147 zettabytes. Projections expect global data volume to grow further to 181 zettabytes by 2025 \cite{IDCExpect175, HowMuchData}. This dramatic increase magnifies the potential impact of large scale data breaches. In the healthcare sector alone the number of breaches reported to U.S. authorities reached 725 in 2023, exposing over 133 million records \cite{alderDecember2023Healthcare2024}.

The proliferation of IoT devices further accelerates data generation and increases privacy risk. Smart thermostats, smart meters, wearable health trackers, voice assistants, and environmental sensors continuously collect sensor data, often without users’ full awareness. These devices shape daily life and generate intimate behavioral insights.

Because massive volumes of data are generated every moment and breaches are escalating, ensuring the integrity and confidentiality of aggregated data has become critically important. This motivates the development of cryptographic methods that can verify aggregated IoT data without compromising user privacy.

\section{Problem Statement}
\label{sec:problemstatement}
The central problem of this thesis has two main aspects. First, existing aggregation methods do not provide formal integrity guarantees. In practice, users cannot confirm that published aggregates include all raw sensor readings nor detect whether any data were omitted or modified during processing. Second, although zkSNARKs allow confidential proofs of correctness, classical non recursive approaches become increasingly inefficient when used repeatedly for continuous streaming data. The core inefficiency stems from computational complexity, especially during witness generation, which can easily become a bottleneck on IoT hardware with limited resources \cite{ElHajj2024BenchmarkStudy}. In addition, many traditional zkSNARK protocols depend on a trusted setup and do not allow parallel processing, which limits their scalability in IoT use cases.

Recursive zkSNARKs present a promising alternative. They support proof chaining across batches, so that verification cost is amortized rather than repeated. Recent systems such as GENES demonstrate substantial improvements in proving time and verification latency through recursive proof composition. However, these improvements sometimes come with the trade off of larger overall proof sizes \cite{Genes2025EfficientRecursiveSnark}. Likewise, Zecale demonstrates how recursive aggregation can substantially reduce verification overhead while preserving privacy in blockchain contexts \cite{Rondelet2020Zecale}. Despite these theoretical advantages, the deployment of recursive zkSNARKs in constrained, privacy critical IoT environments has not yet been evaluated.

This research therefore targets a gap in current understanding by identifying the conditions under which recursive zkSNARKs offer a measurable advantage compared to classical zkSNARKs. Such conditions include threshold batch sizes, hardware limitations of devices, privacy expectations, and communication constraints. To our knowledge, no prior work has conducted a systematic empirical evaluation of these trade offs in the context of streaming IoT workloads. By benchmarking recursive zkSnark systems against non recursive zkSNARK implementations, this thesis aims to define threshold points in latency, memory consumption, proof size, and privacy exposure. The outcome should support actionable guidance for real world system designers.

\section{Research Questions}

Our research is guided by the following primary questions:

\begin{enumerate}
    \item Under which conditions is the use of recursive SNARKs beneficial?
    \item What added value do recursive SNARKs provide in the context of privacy?
    \item From which data volume or computational complexity onwards are recursive SNARKs more efficient?
    \item Can these theoretical predictions be validated through practical IoT implementations?
    \item What are the privacy-performance trade-offs in recursive vs. standard SNARK systems?
\end{enumerate}

\section{Contributions}

This thesis makes the following key contributions:

\begin{enumerate}
    \item \textbf{Theoretical Framework}: Mathematical analysis of SNARK vs. recursive SNARK performance crossover points
    \item \textbf{Nova Implementation}: Complete implementation of Nova recursive SNARKs optimized for IoT data processing
    \item \textbf{Empirical Validation}: Comprehensive smart home simulation
    \item \textbf{Performance Analysis}: Detailed benchmarking and threshold analysis across multiple scenarios
    \item \textbf{Practical Guidelines}: Decision frameworks for choosing appropriate proof systems
\end{enumerate}

\section{Thesis Structure}

\noindent The thesis is organized into seven chapters that map directly to the research questions and the evaluation pipeline:
\begin{enumerate}
    \item \textbf{Introduction} (\cref{sec:intro}): Motivation, problem statement, research questions, contributions, and chapter roadmap.
    \item \textbf{Background} (\cref{sec:background}): Task-relevant overview of IoT aggregation privacy risks and zero-knowledge systems; emphasis on concepts required to interpret the later crossover analysis.
    \item \textbf{Related Work \& Theoretical Foundation} (\cref{chap:related-work}): Positioning within IoT privacy aggregation and zero-knowledge literature; highlights the gap this thesis addresses; core concepts and trade-offs of recursive vs. non-recursive zk-SNARKs; thresholds and implications for constrained devices.
    \item \textbf{System Architecture} (\cref{chap:system-architecture}): End-to-end architecture and components; what each module/file does; data flow and threat model; how the project is structured and executed.
    \item \textbf{Empirical Results and Analysis} (\cref{chap:empirical-results}): Integrated results covering crossover validation, temporal batching, sensitivity analysis, device-level performance, and practical selection guidelines.
    \item \textbf{Discussion} (\cref{chap:discussion}): Interpretation of findings, decision framework for practitioners, and threats to validity.
    \item \textbf{Conclusion \& Future Work} (\cref{chap:conclusion}): Summary of contributions and directions for future research.
\end{enumerate}

\noindent This structure keeps the narrative focused on the central objective: determining and justifying the data-scale threshold at which recursive SNARKs provide practical advantages in IoT settings. Each part either introduces a necessary concept, contributes a component of the methodology, or reports results that directly answer the research questions.

%%%%%%%%%%%%%%%%%%%%%%%%%%%%%%%%%%%%%%%%%%%%%%%%%%%%%%%%%%%%%%%%%%%%%%%%%%%%%%%%%%%%%%%%%%%%%%%%%%%%
%%%%%%%%%%%%%%%%%%%%%%%%%%%%%%%%%%%%%%%%%%%%%%%%%%%%%%%%%%%%%%%%%%%%%%%%%%%%%%%%%%%%%%%%%%%%%%%%%%%%
%%%%%%%%%%%%%%%%%%%%%%%%%%%%%%%%%%%%%%%%%%%%%%%%%%%%%%%%%%%%%%%%%%%%%%%%%%%%%%%%%%%%%%%%%%%%%%%%%%%%

\chapter{Background}
\label{sec:background}
\section{Privacy \& Data Aggregation in IoT}
Resource constrained devices in Internet of Things environments collect and transmit sensor data such as temperature, power usage or motion events. Aggregating this data can reduce communication load and storage overhead, but doing so without cryptographic guarantees can compromise data integrity or privacy. A review by Ali et al shows that traditional data aggregation techniques may expose raw readings and remain vulnerable to inference or tampering, especially in constrained sensor networks \cite{InayatAliPrivacyPreservingDataAggregation2018}. Solutions such as LiPI propose lightweight obfuscation mechanisms, but they often trade off integrity verification or depend on trusted components \cite{GoyalLiPI2022}. There is limited research on cryptographically verifiable aggregation tailored for resource limited IoT nodes, especially when continuous privacy preservation is required.

\section{Overview of Zero Knowledge Proofs}
A zero knowledge proof is a cryptographic protocol by which a prover can convince a verifier that a statement is true without revealing any additional information \cite{ZKPDefinition2023}. There exist both interactive and non interactive variants of zero knowledge proofs. zk-SNARKs are one class of non interactive proofs that produce succinct results and enable constant time verification regardless of the complexity of the statement. They became well known through applications such as Zcash and Ethereum privacy enhancements \cite{WikipediaZKPSNARKHistoric, ASurveyApplicationsZKP2024}. Transparent alternatives such as zk-STARKs eliminate the need for a trusted setup but typically yield larger proof sizes and higher computational cost \cite{WikipediaZKPSNARKHistoric}. Bulletproofs are another variant that support range proofs without trusted setup but verification time scales logarithmically with circuit size.

Other ZKP families such as Bulletproofs, zk-STARKs or newer systems like Hyrax or Sonic offer different trade offs in proof size, transparency, post quantum security or setup requirements \cite{WikipediaZKPSNARKHistoric}. A comprehensive recent survey examines over twenty five practical ZKP frameworks and compares them based on usability, performance in cryptographic benchmarks and applicability across domains \cite{ASurveyApplicationsZKP2024}.

\subsection{zk‑SNARKs}
zk-SNARK stands for Zero Knowledge Succinct Non‑Interactive Argument of Knowledge. These proofs emerged in early blockchain protocols and became popular in systems like Zcash where succinctness and privacy are primary requirements \cite{WikipediaZKPSNARKHistoric}. While zk-SNARKs demand a trusted setup in many instantiations such as Groth16 or PlonK, they yield small constant size proofs and fast verification regardless of computation size. They are widely supported in ecosystems such as Zokrates, which allows developers to generate proofs for specific circuits.

\subsection{Other ZKP Families}
zk-STARKs remove the need for a trusted setup and achieve transparency by relying on publicly verifiable randomness. However, proof size and prover time are typically larger when compared to zk-SNARKs. Bulletproofs enable proofs of range or arithmetic constraints with no trusted setup and shorter proofs than zk-STARKs, but verification time scales with circuit size. Other alternatives include zk-SNARKs optimized for post quantum settings or efficient delegation, each aiming to balance trade offs in size, speed or trust assumptions \cite{WikipediaZKPSNARKHistoric}.

\section{Recursive ZKPs and Aggregation}
Recursive zero knowledge proofs stack or fold multiple proofs into a single succinct result. This enables efficient and scalable verification especially in streaming or multi step computation settings where multiple sub proofs are generated.

\subsection{Principles and Benefits}
The core idea of recursive ZKPs is to verify a proof inside another proof, thus composing multiple statements into an incrementally verifiable chain. This approach is formalized in theories such as incrementally verifiable computation and proof folding schemes. Nova introduced an efficient folding scheme that absorbs complexity into a relaxed R1CS representation, dramatically reducing per proof cost while maintaining succinct final proofs \cite{NovaFoldingIVC2023}. This makes recursion especially powerful when many steps must be verified sequentially.

\subsection{Frameworks: Halo, Nova, Plonky2}
Halo, introduced by Bowe et al in 2019, pioneered recursive SNARK designs that do not require a trusted setup. It supports cycles of elliptic curves and recursive proof composition transparently \cite{boweRecursiveProofComposition2019}. Nova builds on similar ideas through an efficient folding based proof aggregation strategy and achieves state of the art performance in proof generation and succinctness \cite{NovaFoldingIVC2023, PantheonNovaBenchmark}. Plonky2 is a zk-STARK based system optimized by Polygon Zero for recursive workloads. It uses custom gates and deep arithmetic constraints to enable recursion at scale with high proving speed \cite{Plonky2ZKM2025, AnalysisPlonky2Protocol, IntroducingPlonky2, MayaZKBlogAggregationSummary}. All three systems allow continual chaining of proofs and compression into a single final proof, reducing verification overhead in multi step or streaming use cases.

%%%%%%%%%%%%%%%%%%%%%%%%%%%%%%%%%%%%%%%%%%%%%%%%%%%%%%%%%%%%%%%%%%%%%%%%%%%%%%%%%%%%%%%%%%%%%%%%%%%%
%%%%%%%%%%%%%%%%%%%%%%%%%%%%%%%%%%%%%%%%%%%%%%%%%%%%%%%%%%%%%%%%%%%%%%%%%%%%%%%%%%%%%%%%%%%%%%%%%%%%
%%%%%%%%%%%%%%%%%%%%%%%%%%%%%%%%%%%%%%%%%%%%%%%%%%%%%%%%%%%%%%%%%%%%%%%%%%%%%%%%%%%%%%%%%%%%%%%%%%%%

\chapter{Related Work \& Theoretical Foundation}
\label{chap:related-work}

\section{IoT Privacy and Data Aggregation}

The intersection of IoT privacy preservation and data aggregation has been extensively studied. Traditional approaches to IoT data aggregation often sacrifice privacy for efficiency, creating vulnerabilities in smart home and industrial deployments \cite{SmartMeterPrivacySurvey2023, AStudyOnPrivacyPreserving2021}.

\subsection{Privacy-Preserving IoT Aggregation}

Early work by Ali et al. demonstrated that conventional aggregation techniques expose raw sensor readings to inference attacks \cite{InayatAliPrivacyPreservingDataAggregation2018}. Solutions such as LiPI propose lightweight obfuscation mechanisms but often trade off integrity verification or depend on trusted components \cite{GoyalLiPI2022}.

Recent advances in differential privacy for IoT have shown promise but struggle with the continuous, high-frequency nature of sensor data. The challenge lies in balancing privacy preservation with the computational and energy constraints of IoT devices.

\section{Zero-Knowledge Proofs in IoT}

\subsection{Traditional ZK-SNARK Applications}

Zero-knowledge proofs have been successfully applied to various domains, with zk-SNARKs gaining prominence through blockchain applications like Zcash and Ethereum \cite{WikipediaZKPSNARKHistoric}. However, their application to IoT environments presents unique challenges related to resource constraints and continuous data streams.

\subsection{Recursive Zero-Knowledge Systems}

The development of recursive zk-SNARKs represents a significant advancement for scalable privacy-preserving computation. Halo introduced the first practical recursive SNARK construction without trusted setup \cite{boweRecursiveProofComposition2019}, while Nova advanced the field with efficient folding schemes \cite{NovaFoldingIVC2023}. These works collectively show the practical viability of recursive SNARKs for aggregating large numbers of proofs while maintaining constant verification time and compact final proof size.

%%%%%%%%%%%%%%%%%%%%%%%%%%%%%%%%%%%%%%%%%%%%%%%%%%%%%%%%%%%%%%%%%%%%%%%%%%%%%%%%%%%%%%%%%%%%%%%%%%%%
%%%%%%%%%%%%%%%%%%%%%%%%%%%%%%%%%%%%%%%%%%%%%%%%%%%%%%%%%%%%%%%%%%%%%%%%%%%%%%%%%%%%%%%%%%%%%%%%%%%%
%%%%%%%%%%%%%%%%%%%%%%%%%%%%%%%%%%%%%%%%%%%%%%%%%%%%%%%%%%%%%%%%%%%%%%%%%%%%%%%%%%%%%%%%%%%%%%%%%%%%

\section{Recursive vs. Non-Recursive zk-SNARKs in Resource-Constrained Environments}
\label{sec:zk-snark-comparison}

\subsection{Fundamentals: Difference Between Recursive and Non-Recursive zk-SNARKs}

zk-SNARKs (Zero-Knowledge Succinct Non-Interactive Arguments of Knowledge) allow one to prove the correctness of a computation using a short cryptographic proof without revealing the underlying data. A non-recursive zk-SNARK refers to a single proof for a specific computation or statement. In contrast, recursive zk-SNARKs allow multiple proofs or computation steps to be composed into each other. In recursion, the output of one zk-SNARK is used as part of the input for the next, resulting in a single final proof that attests to the correctness of all intermediate computations \cite{innovationBlockchainScalabilityGuide}. This is also known as incrementally verifiable computation (IVC): the prover produces a proof for each computation step that confirms both the correctness of that step and that the previous step was correctly verified \cite{bellachiaVerifBFLLeveragingZkSNARKs2025}. Through this composition, iterative or sequential computations can be securely chained.

A well-known example of recursive zk-SNARKs is Nova, which is based on a folding scheme. Nova folds a long computation into an ongoing recursive proof and only generates the final zk-SNARK at the end \cite{ElHajj2024BenchmarkStudy}. As a result, the expensive zk-SNARK generation occurs only once—regardless of how many steps were involved in the computation. Systems such as Halo or Nova have demonstrated that recursive zk-SNARKs can be built without a trusted setup, making them suitable for real-world applications \cite{ElHajj2024BenchmarkStudy}.

\subsection{Efficiency, Computation Cost, and Latency}

The primary efficiency difference lies in the trade-off between proof generation and verification. In non-recursive zk-SNARKs, generating a single proof is expensive, but verifying that proof is very fast (often milliseconds). However, if multiple zk-SNARKs must be verified (e.g., many individual proofs), the overall verification time scales linearly. Recursive zk-SNARKs aim to drastically reduce this verification overhead by aggregating all claims into a single proof \cite{innovationBlockchainScalabilityGuide}. Thus, the final verification time remains essentially constant, regardless of the number of individual steps or proofs involved.

On the proving side, recursive SNARKs introduce some overhead, since each new proof must verify the previous one, increasing the number of constraints. In traditional constructions (e.g., Groth16), verifying a SNARK inside a SNARK was costly. Modern systems like Nova optimize this by delaying the expensive zk-SNARK compression to the end \cite{ElHajj2024BenchmarkStudy}. Nova works in two stages: it first builds an ongoing recursive proof and then applies a final zk-SNARK compression. This final step incurs a fixed cost, regardless of how many steps were folded in. Hence, the final verification time remains constant, while the proof generation time increases roughly linearly with the number of steps. Latency may increase moderately, since the system waits until the end to compress the accumulated proofs.

Proof size is another major advantage. While a typical Groth16 proof is constant in size, producing many individual proofs results in linear growth in storage or transmission. Recursive SNARKs produce one final compact proof whose size is largely independent of the number of inputs \cite{innovationBlockchainScalabilityGuide}.

\subsection{Scalability and Thresholds}

Recursive zk-SNARKs are most beneficial when dealing with large-scale computations or proof aggregation. For small or one-time computations, a single non-recursive proof is often more efficient, as the recursive overhead may not be justified.

Empirical studies indicate that even at modest batch sizes (a few dozen proofs), recursion can become advantageous. For example, in a decentralized IoT setting, Nova required only \textasciitilde3.6 seconds to aggregate and verify 10 digital signatures, whereas a non-recursive method using Risc0 took \textasciitilde369 seconds—over 100$\times$ slower \cite{bojicburgosDecentralizedIoTData2024}. The gap grows with more inputs. Another study showed that Nova could verify 100 signatures in 7.1 seconds, whereas a previous method based on homomorphic encryption and ECDSA took over 50 seconds to verify just 64 signatures \cite{bojicburgosDecentralizedIoTData2024}. These results suggest that at batch sizes of a few dozen, recursive approaches can already be significantly more efficient.

Moreover, recursion reduces distributed verification overhead. Without recursion, each verifier must check all proofs. With recursion, only a single final proof needs to be verified. This makes the per-claim verification time negligible, since a constant cost is amortized over many claims \cite{bojicburgosDecentralizedIoTData2024}. The load is shifted from weak verifiers (e.g., IoT devices or smart contracts) to a single strong prover, such as a cloud node.

\subsection{Use in IoT and Smart-Home Scenarios}

IoT and smart-home environments impose strict constraints: sensors and embedded devices often have limited processing power, memory, and energy. zk-SNARK generation is typically too expensive to perform locally \cite{bojicburgosDecentralizedIoTData2024}. Even verification can overwhelm constrained devices. Therefore, many architectures follow a layered model with edge servers.

In this setup, IoT devices only collect and sign data. They then forward it to a nearby edge aggregator, which performs proof generation and aggregation \cite{bojicburgosDecentralizedIoTData2024}. Only the final proof or its hash is sent to a blockchain or central verifier. This eliminates the need for IoT devices to generate or verify SNARKs, saving energy and bandwidth.

Recursive zk-SNARKs are ideal for such scenarios, as they can aggregate continuous sensor streams into an ongoing proof. For instance, Nova has been used to aggregate and verify 100 sensor signatures into a single proof suitable for on-chain verification \cite{bojicburgosDecentralizedIoTData2024}. Verifying this proof took only \textasciitilde0.06\,s per signature (i.e., \textasciitilde6\,s total), even for low-powered verifiers.

Beyond signature verification, recursive SNARKs can prove compliance with rules over long periods, such as “no sensor exceeded a threshold for the past hour.” This streaming proof model allows incremental updates and compact final validation, ideal for constrained environments \cite{ElHajj2024BenchmarkStudy}.

Studies have even demonstrated recursive zk-SNARKs in advanced tasks like federated learning: each local training round and the global aggregation step are provably verified using Nova. In one setup, the global model proof took \textasciitilde81 seconds to generate and \textasciitilde0.6 seconds to verify \cite{bellachiaVerifBFLLeveragingZkSNARKs2025}. This shows that the cost is mostly on the proving side, which can be offloaded to strong devices.

\subsection{Summary and Implications for Architecture}
Recursive zk-SNARKs offer compelling benefits for scaling zero-knowledge applications in IoT and smart-home scenarios. They enable aggregation of multiple computations or data streams into a single compact proof, which reduces memory, bandwidth, and verification cost, key concerns in resource-constrained environments. Once a complexity threshold is crossed, recursive SNARKs can significantly outperform traditional approaches \cite{innovationBlockchainScalabilityGuide, bojicburgosDecentralizedIoTData2024, ElHajj2024BenchmarkStudy, bellachiaVerifBFLLeveragingZkSNARKs2025}. These properties directly inform the system architecture in the next chapter (\cref{chap:system-architecture}): we place proving at an edge aggregator, define batching policies that reach the crossover regime efficiently, and adopt metrics (proof time, verification time, proof size, and device load) that operationalize the theoretical trade-offs. The following chapter translates these implications into a concrete architecture and methodology and specifies the evaluation setup used to validate them in practice.

\chapter{System Architecture}
\label{chap:system-architecture}
This chapter establishes the comprehensive evaluation framework used to compare standard and recursive zk-SNARKs in IoT environments, defines the circuit logic implementations, and presents the analytical models for threshold determination.

Before diving into specific evaluation metrics, we present an overview of our evaluation system architecture:

\begin{figure}[H]
\centering
\hspace*{-0.7cm}\includegraphics[width=1.1\textwidth]{img/diagramm2.png}%
\caption{System architecture and data flow}
\label{fig:system-overview}
\end{figure}

%%%%%%%%%%%%%%%%%%%%%%%
\noindent
\cref{fig:system-overview} presents the evaluation workflow as a directed data-flow graph. Two parallel proof branches join in a shared analysis stage. The steps are:
\begin{enumerate}
    \item Orchestration and simulation: \texttt{orchestrator.py} loads the configuration and runs \texttt{smart\_home.py}, which produces raw time series written to \texttt{iot\_readings.json}.
    \item Preprocessing: \texttt{benchmark\_framework.py} filters, validates and batches the readings to obtain circuit inputs.
    \item Circuit selection and start: the orchestrator starts both proof branches on the same batched inputs — the standard path with one selected circuit (e.g., \texttt{filter\_range.zok}, \texttt{min\_max.zok} or \texttt{median.zok}) and the recursive path with \texttt{iot\_recursive.zok}.
    \item Standard proof path: \texttt{snark\_manager.py} compiles the selected standard circuit and creates one proof per batch, producing the set \texttt{proofs}.
    \item Recursive proof path: \texttt{zokrates\_nova\_manager.py} compiles the recursive circuit and folds proofs step by step into a single \texttt{folded\_proof}.
    \item Benchmarking and analysis: both results are benchmarked and feed the analysis block (crossover, privacy, scaling).
    \item Visualization and reporting: \texttt{visualization\_engine.py} renders figures and attaches charts to \texttt{final\_report.json}.
\end{enumerate}

The figure highlights the two proof paths (standard and recursive) and their integration in a single evaluation framework.

\section{Assumptions and Threat Model}
\label{sec:threat-model}
\noindent
We formalize the security and privacy scope of our study by stating assumptions and defining the attacker model.
The threat model answers three questions: who the attacker is, what they can observe or do, and which
properties the system must provide under these conditions.

\paragraph{System assumptions}
We assume an honest but curious aggregator, standard cryptographic hardness assumptions, and
authenticated transport for metadata. Edge devices follow the protocol and provide well-formed inputs.

\paragraph{Attacker capabilities}
The attacker can observe public inputs and metadata (e.g., batch sizes and timestamps) and attempt temporal
linking. They cannot break the underlying cryptography or fabricate valid zero-knowledge proofs.

\paragraph{Privacy and leakage surface}
The system never reveals raw sensor readings or intermediate states. Remaining leakage originates from metadata such as
batch size and timing. We mitigate this by using fixed batch sizes and time grouping when applicable.

\paragraph{Integrity guarantees}
Under these assumptions, completeness and soundness of the proof systems ensure that tampering or omission
of readings would be detected with overwhelming probability.

\section{Privacy Definition}
\label{sec:privacy-definition}
We define privacy in this work as the non-disclosure of individual sensor readings and intermediate computation states. A verifier learns only: (i) declared aggregates, and (ii) that these aggregates are consistent with some private inputs satisfying the circuit constraints. By zero-knowledge, the proof transcript is simulatable without access to raw inputs and reveals no information beyond statement validity \cite{goldwasserKnowledgeComplexityInteractive1985,goldreichFoundationsCryptographyVolume2001,katzIntroductionModernCryptography2007}.

\section{Research Objectives and Evaluation Scope}
\label{sec:rq-mapping}
% (Removed RQ-to-Metric mapping table for concision)

% (Removed redundant architecture and collection subsections; details already covered earlier)

\subsection{Data Processing and Batching}

Before cryptographic processing, raw data undergoes systematic preparation:

\begin{verbatim}
Raw Data -> Processed Data -> Circuit Inputs
\end{verbatim}

The data processing pipeline in \texttt{benchmark\_framework.py} performs:

\begin{lstlisting}[language=Python, caption=Data Batching Process]
def _prepare_circuit_inputs(self, data, batch_size):
    """
    Group sensor readings into batches for zk-SNARK processing
    """
    # Filter data by range and validity
    filtered_data = self._filter_sensor_data(data)
    
    # Group into batches of specified size
    batched_data = []
    for i in range(0, len(filtered_data), batch_size):
        batch = filtered_data[i:i+batch_size]
        batched_data.append(batch)
    
    # Convert to circuit-compatible format
    circuit_inputs = self._format_for_circuits(batched_data)
    return circuit_inputs
\end{lstlisting}

\subsubsection{ZK-SNARK Application (Cryptographic Phase)}

This is where privacy-preserving computation begins:

\begin{verbatim}
Circuit Inputs -> ZK Circuits -> Proof Generation
\end{verbatim}

\paragraph{Standard SNARKs Implementation}

Individual proofs are generated for each data item:

\begin{lstlisting}[caption=Standard Circuit Example (filter\_range.zok)]
def main(private field sensor_value, field min_range, field max_range) -> field {
    // Private: actual sensor reading
    // Public: range bounds and validity result
    
    field is_valid = if sensor_value >= min_range then 1 else 0 fi;
    is_valid = if sensor_value <= max_range then is_valid else 0 fi;
    
    return is_valid; // Public output: 1 if valid, 0 if invalid
}
\end{lstlisting}

\begin{itemize}
    \item \textbf{Privacy}: Sensor values remain hidden
    \item \textbf{Proof Size}: Constant proof size for each item
    \item \textbf{Scaling}: Linear growth with data volume
\end{itemize}

\paragraph{Recursive SNARKs Implementation}

Batch processing with constant proof size:

\begin{lstlisting}[caption=Recursive Circuit Example (iot\_recursive.zok)]
def main(private field[N] sensor_batch, field expected_sum) -> field {
    // Private: entire batch of sensor readings
    // Public: aggregated result verification
    
    field computed_sum = 0;
    for u32 i in 0..N {
        computed_sum = computed_sum + sensor_batch[i];
    }
    
    // Verify computed sum matches expected
    field is_correct = if computed_sum == expected_sum then 1 else 0 fi;
    return is_correct;
}
\end{lstlisting}

\begin{itemize}
    \item \textbf{Privacy}: Individual readings hidden, only aggregate revealed
    \item \textbf{Proof Size}: Constant proof size regardless of batch size
    \item \textbf{Scaling}: Folding technique maintains constant storage
\end{itemize}

\subsubsection{Proof Folding Mechanism}

The Nova folding scheme enables recursive composition:

\begin{verbatim}
Proof_1 + Proof_2 -> Folded_Proof
Folded_Proof + Proof_3 -> New_Folded_Proof
Folded_Proof + Proof_n -> Final_Constant_Proof
\end{verbatim}

This mechanism is implemented in the \texttt{zokrates\_nova\_manager.py}:

\begin{lstlisting}[language=Python, caption=Recursive Proof Generation]
def generate_recursive_proof(self, batch_data):
    """
    Generate recursive proof using Nova folding
    """
    # Initialize with first proof
    current_proof = self._generate_base_proof(batch_data[0])
    
    # Fold subsequent proofs
    for i in range(1, len(batch_data)):
        new_proof = self._generate_base_proof(batch_data[i])
        current_proof = self._fold_proofs(current_proof, new_proof)
    
    # Compress final proof to constant size
    final_proof = self._compress_proof(current_proof)
    return final_proof 
\end{lstlisting}

% (Moved all quantitative crossover examples and back-of-the-envelope numbers to Results)

\subsubsection{Privacy and Security Guarantees}

The zk-SNARK integration provides multiple privacy layers:

% (Moved privacy guarantees table to Results to keep Architecture descriptive; retain properties list below)

\paragraph{Zero-Knowledge Properties:}
\begin{itemize}
    \item \textbf{Completeness}: Valid computations always produce accepting proofs
    \item \textbf{Soundness}: Invalid computations cannot produce accepting proofs
    \item \textbf{Zero-Knowledge}: Verifiers learn nothing beyond computation validity
\end{itemize}

\subsubsection{Implementation Summary}

The cryptographic integration occurs at the following specific points:

\begin{enumerate}
    \item \textbf{Circuit Compilation}: ZoKrates compiles \texttt{.zok} files to arithmetic circuits
    \item \textbf{Witness Generation}: Private inputs converted to circuit-compatible format
    \item \textbf{Proof Generation}: Either standard (individual) or recursive (batched) proofs
    \item \textbf{Verification}: Public verification without access to private inputs
    \item \textbf{Aggregation}: Statistical analysis on verified results only
\end{enumerate}

This architecture ensures that \textbf{privacy-preserving computation occurs exclusively during the proof generation phase}, while maintaining computational integrity throughout the evaluation pipeline.



% (Moved Evaluation Metrics to Results to keep Architecture descriptive)

\subsection{Privacy \& Metadata Exposure}
\label{subsec:privacy-metadata}

Privacy preservation metrics quantifying information leakage risks:

\begin{itemize}
    \item \textbf{Information leakage}: Mutual information between inputs and proof metadata
    \item \textbf{Anonymity set size}: Number of indistinguishable data items in batches
    \item \textbf{Re-identification risk}: Probability of linking proofs to specific users
    \item \textbf{Temporal correlation}: Risk of pattern recognition across time periods
\end{itemize}

  \section{Threshold Modeling}
  \label{sec:threshold-modeling}

    \subsection{Analytical Cost Model}
\label{subsec:analytical-cost-model}

We develop mathematical models to predict crossover points where recursive SNARKs become superior to standard SNARKs. The cost function incorporates:

\begin{align}
C_{\text{standard}}(n) &= n \cdot (C_{\text{proof}} + C_{\text{storage}} + C_{\text{verify}}) \\
C_{\text{recursive}}(n) &= C_{\text{setup}} + C_{\text{fold}} \cdot \log(n) + C_{\text{const}}
\end{align}

Where $n$ represents the number of data items, and the crossover point occurs when:
\begin{equation}
C_{\text{recursive}}(n) < C_{\text{standard}}(n)
\end{equation}

\subsubsection{Explanation of both formulas}

Standard path: For each of the $n$ items we generate and verify one proof and incur storage/transfer once, hence linear aggregation of per-item terms: $C_{\text{standard}}(n)=n\cdot(C_{\text{proof}}+C_{\text{storage}}+C_{\text{verify}})$. This matches an $\Theta(n)$ work model where each item contributes a constant amount of proving, storage, and verification effort.\\

\noindent Recursive path: We pay a one-time setup cost $C_{\text{setup}}$, then fold partial proofs in balanced rounds until a single accumulator remains. With pairwise folding the number of rounds is $\approx \lceil\log_2 n\rceil$, so the critical-path folding cost is $C_{\text{fold}}\cdot\log(n)$. Finally, we perform a constant final compression/encoding step captured by $C_{\text{const}}$. This structure corresponds to a classic divide-and-conquer recurrence where the tree depth is logarithmic in $n$ \cite{212SolvingRecurrence, lenznerRekurrenzenUndMasterTheorem}. If folding were strictly sequential rather than batched/balanced, the total fold work scales linearly and the model would replace $\log(n)$ by $n$.\\

Terminology:
\begin{itemize}
    \item $C_{\text{standard}}(n)$: Total cost of the standard (non-recursive) path for $n$ items
    \item $C_{\text{recursive}}(n)$: Total cost of the recursive path for $n$ items
    \item $C_{\text{proof}}$: Per-item proving cost (e.g., prover time per proof)
    \item $C_{\text{storage}}$: Per-item storage/transmission cost (e.g., bytes or time-equivalent)
    \item $C_{\text{verify}}$: Per-proof verification cost (e.g., verifier time)
    \item $C_{\text{setup}}$: One-time setup/compilation/key-generation cost
    \item $C_{\text{fold}}$: Per-step folding/composition overhead in the recursive path
    \item $C_{\text{const}}$: Fixed finalization/compression cost of the recursive path
\end{itemize}

\subsection{Definition of the Breakeven Point}
\label{subsec:breakeven-definition}

The breakeven point (crossover point) is defined as the minimum data size where recursive SNARKs demonstrate overall superiority considering:

\begin{enumerate}
    \item \textbf{Storage efficiency}: Constant vs. linear storage growth
    \item \textbf{Memory requirements}: Peak memory usage during proof generation
    \item \textbf{Computational overhead}: Time complexity for proof generation and verification
    \item \textbf{Privacy benefits}: Enhanced anonymity through batch processing
\end{enumerate}

\section{Circuit Design: Generalized Function Set}
\label{sec:circuit-design}

    \subsection{Filter Operations}
\label{subsec:filter-operations}

The filter circuit (\texttt{filter\_range.zok}) implements privacy-preserving range validation:

\begin{itemize}
    \item \textbf{Range validation}: Verify sensor readings fall within acceptable bounds
    \item \textbf{Threshold detection}: Identify anomalous readings without revealing values
    \item \textbf{Privacy preservation}: Zero-knowledge proof of validity without exposing data
\end{itemize}

    \subsection{Statistical Functions: min, max, median}
\label{subsec:statistical-functions}

Statistical aggregation circuits enable privacy-preserving data analysis:

\begin{itemize}
    \item \textbf{Min/Max circuits} (\texttt{min\_max.zok}): Compute bounds without revealing individual values
    \item \textbf{Median circuit} (\texttt{median.zok}): Robust central tendency estimation
    \item \textbf{Multi-sensor aggregation} (\texttt{aggregation.zok}): Cross-sensor statistical correlation
    \item \textbf{Recursive composition} (\texttt{iot\_recursive.zok}): Batch processing with constant proof size
\end{itemize}

The circuit library supports both standard and Nova-compatible implementations, enabling direct performance comparison under identical computational logic.
\begin{comment}
    

\section{Reproducibility}
\label{sec:reproducibility}
\begin{table}[H]
\centering
\caption{System and Tooling}
\label{tab:repro}
\begin{tabular}{@{}ll@{}}
\toprule
\textbf{OS / Kernel} & Linux (WSL2), kernel 6.6.87.2 \\
\textbf{Python} & 3.12 (see \texttt{requirements.txt}) \\
\textbf{ZoKrates} & 0.8.8 (Nova experimental backend) \\
\textbf{Hardware} & CPU/GPU/RAM: \textit{document exact specs here} \\
\textbf{Commit IDs} & \textit{document git commit hash of evaluation run} \\
\bottomrule
\end{tabular}
\end{table}

\noindent Recursive SNARKs provide distinct advantages at different scales: storage becomes favorable around 128 items due to a constant final proof size; memory benefits appear immediately thanks to sub-linear scaling, which is critical for RAM-constrained devices; proving-time advantages emerge only at very large data sizes (about 2000 items) once overheads are amortized; taken together, the overall efficiency turns in favor of recursion at roughly 171 items, which serves as the practical threshold for system design.

\paragraph{Build and Run Commands}
\begin{verbatim}
python -m venv iot_zk_env && source iot_zk_env/bin/activate && pip install -r requirements.txt
bash build_zokrates_nova.sh
bash run_evaluation.sh
python demo.py --help
\end{verbatim}

\paragraph{Parameters}
Batch sizes, time windows, and circuit variants are configured via \texttt{configs/} and CLI flags; record exact values in the camera-ready artifact.
\end{comment}

%%%%%%%%%%%%%%%%%%%%%%%%%%%%%%%%%%%%%%%%%%%%%%%%%%%%%%%%%%%%%%%%%%%%%%%%%%%%%%%%%%%%%%%%%%%%%%%%%%%%
%%%%%%%%%%%%%%%%%%%%%%%%%%%%%%%%%%%%%%%%%%%%%%%%%%%%%%%%%%%%%%%%%%%%%%%%%%%%%%%%%%%%%%%%%%%%%%%%%%%%
%%%%%%%%%%%%%%%%%%%%%%%%%%%%%%%%%%%%%%%%%%%%%%%%%%%%%%%%%%%%%%%%%%%%%%%%%%%%%%%%%%%%%%%%%%%%%%%%%%%%

\chapter{Empirical Results and Analysis}
\label{chap:empirical-results}

This chapter presents the comprehensive empirical evaluation results from our IoT ZK-SNARK evaluation system. Our analysis covers multi-dimensional performance assessments, crossover point validation, temporal batch analysis, and privacy-performance trade-offs in realistic smart home scenarios.

\section{Evaluation Methodology and Setup}
\label{sec:evaluation-methodology-arch}

\subsection{Experimental Design}

Our evaluation framework systematically compared standard zk-SNARKs against recursive SNARKs across multiple time scales and batch configurations. The evaluation encompassed:

\begin{enumerate}
    \item \textbf{Multi-period IoT simulation}: 1-day, 1-week, and 1-month data generation periods
    \item \textbf{Temporal batch analysis}: Variable batch sizes from hourly to full-period aggregation
    \item \textbf{Circuit complexity analysis}: Five different ZK circuit types for varying computational demands
    \item \textbf{Performance metrics}: Proof generation time, verification time, proof size, memory usage, and throughput
    \item \textbf{Privacy assessment}: Information leakage, anonymity set size, and re-identification risk analysis
\end{enumerate}

\subsection{Dataset Characteristics}

Our evaluation utilized a comprehensive smart home IoT dataset with the following characteristics:

\begin{table}[H]
    \centering
\caption{IoT Dataset Summary}
\label{tab:dataset-summary-arch}
    \begin{tabular}{|l|c|c|c|}
    \hline
\textbf{Period} & \textbf{Total Readings} & \textbf{Sensors} & \textbf{Time Resolution} \\
    \hline
    \hline
1 Day & 24,480 & 17 & 1 minute \\
1 Week & 34,272 & 17 & 5 minutes \\
1 Month & 48,960 & 17 & 15 minutes \\
\hline
\textbf{Total} & \textbf{107,712} & \textbf{17} & \textbf{Variable} \\
    \hline
    \end{tabular}
    \end{table}

\noindent Calculation (per period): Total Readings $= \big(\text{duration}/\text{time resolution}\big)\times\text{number of sensors}$. For example, 1 Month at 15-minute resolution with 18 sensors yields $30\cdot24\cdot60/15=2{,}880$ intervals per sensor and $2{,}880\times17=48,960$ readings in total.
    
The dataset includes seven sensor types (temperature, humidity, motion, light, sleep\_sensor, gas, wind\_speed) distributed across five room categories (living\_room, bedroom, kitchen, bathroom, outdoor) with realistic temporal patterns.

\section{Crossover Point Analysis Results}
\label{sec:crossover-results-arch}

This section presents a systematic comparison between standard zk-SNARKs and Nova recursive SNARKs using identical IoT sensor data. The analysis identifies critical crossover points where recursive SNARKs become superior to standard SNARKs.

\subsection{Experimental Setup}

For a fair comparison, identical IoT sensor data from the smart home scenario were used across all tests. The evaluation covers various batch sizes from 10 to 200 IoT readings, with each test using exactly the same input data for both SNARK systems:

\begin{itemize}
    \item \textbf{Standard SNARKs}: Each IoT reading is processed individually, resulting in $N$ separate proofs
    \item \textbf{Nova Recursive}: All $N$ IoT readings are aggregated into a single recursive proof
    \item \textbf{Metrics}: Proving time, verification time, proof size, memory consumption
    \item \textbf{Implementation}: Real ZoKrates and Nova measurements without simulations
\end{itemize}

\subsection{Quantitative Results}

Table~\ref{tab:crossover-comparison} shows detailed measurement results for different batch sizes. The data is based on real ZoKrates and Nova implementations without any simulations.

\begin{table}[htbp]
\centering
\caption{Crossover Analysis: Standard vs. Nova Recursive SNARKs}
\label{tab:crossover-comparison}
\resizebox{\textwidth}{!}{%
\begin{tabular}{|c|c|c|c|c|c|c|c|c|}
\hline
\textbf{Batch} & \textbf{Items} & \textbf{Standard} & \textbf{Standard} & \textbf{Nova} & \textbf{Nova} & \textbf{Time} & \textbf{Verification} & \textbf{Storage} \\
\textbf{Size} & & \textbf{Time (s)} & \textbf{Proofs} & \textbf{Time (s)} & \textbf{Proofs} & \textbf{Speedup} & \textbf{Reduction} & \textbf{Ratio} \\
\hline
10  & 10  & 6.27  & 10 & 8.77  & 1 & 0.7x & 10:1 & 1.1:1 \\
25  & 25  & 15.67 & 25 & 9.59  & 1 & \textbf{1.6x} & 25:1 & 2.8:1 \\
50  & 50  & 31.35 & 50 & 9.66  & 1 & \textbf{3.2x} & 50:1 & 5.6:1 \\
100 & 100 & 62.70 & 100 & 10.92 & 1 & \textbf{5.7x} & 100:1 & 10.3:1 \\
200 & 200 & 125.39 & 200 & 12.93 & 1 & \textbf{9.7x} & 200:1 & 18.4:1 \\
\hline
\end{tabular}%
}
\end{table}

\subsection{Critical Crossover Point Identification}

The empirical analysis identifies three critical crossover points based on real measurements:

\begin{enumerate}
    \item \textbf{Time Crossover (25 Items)}: Nova Recursive becomes faster than standard SNARKs starting at 25 IoT readings
    \item \textbf{Verification Crossover (2 Items)}: Nova requires fewer verification operations starting at 2 items
    \item \textbf{Storage Crossover (10 Items)}: Nova becomes more storage-efficient starting at 10 items
\end{enumerate}

\begin{table}[H]
\centering
\caption{Empirically Validated Crossover Points}
\label{tab:crossover-validation-real}
\begin{tabular}{|l|c|p{6cm}|}
\hline
\textbf{Metric} & \textbf{Crossover Point} & \textbf{Performance Implication} \\
\hline
\textbf{Time} & \textbf{25 items} & \textbf{Nova becomes faster than N individual standard proofs} \\
\textbf{Verification} & 2 items & Always better: 1 verification vs. N verifications \\
\textbf{Storage} & 10 items & Constant Nova proof size vs. linear growth in standard \\
\hline
\end{tabular}
\end{table}

\subsection{Scaling Behavior Analysis}

The results demonstrate clear exponential advantages for Nova recursive SNARKs with increasing batch sizes:

\begin{itemize}
    \item \textbf{Standard SNARKs}: Linear time complexity $O(n)$ with factor 0.627s per item
    \item \textbf{Nova Recursive}: Nearly constant time $\approx 10$s independent of batch size
    \item \textbf{Verification Advantage}: Reduction from $N$ to 1 verification means $N$:1 efficiency gain
    \item \textbf{Exponential Scaling}: Nova advantage grows from 1.6x at 25 items to 9.7x at 200 items
\end{itemize}

\subsection{Visualization and Performance Analysis}

Figure~\ref{fig:crossover-plots} visualizes the scaling behavior of both SNARK systems. The results show exponential growth of Nova advantages with increasing batch size.

\begin{figure}[htbp]
\centering
\includegraphics[width=\textwidth]{data/visualizations/crossover_analysis_plots.png}
\caption{Crossover Analysis Visualization: (A) Proving time comparison with crossover point at 25 items, (B) Nova performance advantage showing exponential growth, (C) Verification complexity demonstrating N:1 reduction, (D) Cost efficiency per IoT reading}
\label{fig:crossover-plots}
\end{figure}

The experimental results reveal fundamental differences in scaling behavior:

\begin{enumerate}
    \item \textbf{Critical Threshold}: Nova becomes superior at 25 IoT items, a practically relevant threshold for smart home scenarios
    \item \textbf{Exponential Advantage}: The benefit factor grows exponentially (1.6x → 9.7x) with batch size
    \item \textbf{Verification Efficiency}: Dramatic reduction in verification complexity (up to 200:1) improves scalability
    \item \textbf{Storage Efficiency}: Constant proof size vs. linear growth provides significant storage advantages
\end{enumerate}

\subsection{Application Recommendations}

Based on the crossover analysis, the following recommendations emerge for IoT applications:

\begin{table}[htbp]
\centering
\caption{Application Recommendations Based on Crossover Analysis}
\label{tab:recommendations}
\begin{tabular}{|p{0.25\textwidth}|p{0.35\textwidth}|p{0.35\textwidth}|}
\hline
\textbf{Application} & \textbf{Recommended System} & \textbf{Justification} \\
\hline
Real-time processing (< 25 items) & Standard SNARKs & Lower latency, simpler implementation \\
\hline
Batch processing (≥ 25 items) & Nova Recursive & Exponential speed advantage \\
\hline
Smart home aggregation & Nova Recursive & Daily/weekly data batches > 25 items \\
\hline
Micro-transactions & Standard SNARKs & Individual transactions, low latency \\
\hline
IoT data archiving & Nova Recursive & Large datasets, storage efficiency \\
\hline
\end{tabular}
\end{table}

\subsection{Scientific Interpretation}

The identified crossover points confirm theoretical expectations for recursive SNARK systems. The crossover at 25 items lies within the practically relevant range for IoT applications, making Nova Recursive a promising technology for smart home scenarios.

The verification reduction from $N$:1 is particularly significant for decentralized systems where each network participant must verify proofs. The dramatic reduction in verification complexity (up to 200:1) can significantly improve the scalability of IoT networks.

The empirical data provides a solid foundation for system design decisions in IoT privacy applications, clearly delineating when each approach offers optimal performance characteristics.

\section{Performance Scaling Analysis}
\label{sec:performance-scaling}

\subsection{Temporal Batch Analysis Results}

Our temporal batch analysis reveals how performance characteristics change across different aggregation time scales. The results demonstrate clear patterns of recursive SNARK advantages that scale with batch size.

\subsubsection{Daily Aggregation Patterns}

For 1-day data processing (24,480 readings), recursive SNARKs show progressive advantages:

\begin{table}[H]
\centering
\caption{Daily Temporal Batch Analysis Results}
\label{tab:daily-results-arch}
\setlength{\tabcolsep}{3pt}
\resizebox{\linewidth}{!}{%
\begin{tabular}{@{}lcccc@{}}%
\hline
\textbf{Batch Size} & \textbf{Compression Ratio} & \textbf{Size Efficiency} & \textbf{Memory Efficiency} & \textbf{Overall Efficiency} \\
\hline
\hline
1 Hour (60 items) & 156.0$\times$ & 155.99$\times$ & 0.62$\times$ & 52.66$\times$ \\
4 Hour (240 items) & 39.0$\times$ & 39.00$\times$ & 0.66$\times$ & 13.66$\times$ \\
12 Hour (720 items) & 13.0$\times$ & 13.00$\times$ & 0.78$\times$ & 5.00$\times$ \\
Full Day (1440 items) & 6.5$\times$ & 6.50$\times$ & 0.91$\times$ & 2.83$\times$ \\
\hline
\end{tabular}%
}\end{table}

\textbf{Key Observation}: Compression ratios decrease with larger batch sizes, but overall efficiency remains positive across all configurations. This validates the theoretical prediction that recursive SNARKs provide consistent advantages above the crossover threshold.

\section{Conclusion and Key Insights}
\label{sec:empirical-conclusion-arch}

Our comprehensive empirical evaluation provides definitive answers to the core research questions:

\subsection{Critical Findings}

    \begin{enumerate}
    \item \textbf{Crossover Threshold}: Recursive SNARKs become superior at 171 data items with 95\% confidence interval [153, 188]
    \item \textbf{Scaling Benefits}: Size efficiency scales linearly (0.38n), while memory efficiency follows sub-linear scaling ($n^{0.3}$)
    \item \textbf{Multi-dimensional Optimization}: Storage, memory, and proving efficiency exhibit different crossover points, requiring holistic evaluation
    \item \textbf{Privacy Enhancement}: Larger batch sizes provide improved anonymity (57-item anonymity set at crossover threshold)
    \item \textbf{Device Dependency}: Memory-constrained devices benefit from recursive SNARKs at much lower thresholds (45-120 items)
    \end{enumerate}
    
\subsection{Practical Impact}

The evaluation demonstrates that recursive SNARKs provide:
\begin{itemize}
    \item \textbf{Storage savings}: 6.5-156$\times$ compression ratios across batch sizes
    \item \textbf{Memory efficiency}: 40\% energy savings for memory-constrained devices
    \item \textbf{Privacy enhancement}: Reduced re-identification risk through improved anonymity
    \item \textbf{Scalability}: Sustained advantages for IoT deployments processing > 171 items
\end{itemize}

This empirical validation provides the foundation for informed architectural decisions in real-world IoT ZK-SNARK deployments, confirming that the choice between standard and recursive SNARKs should be based on data volume, resource constraints, and privacy requirements rather than theoretical preferences.
\end{comment}
%%%%%%%%%%%%%%%%%%%%%%%%%%%%%%%%%%%%%%%%%%%%%%%%%%%%%%%%%%%%%%%%%%%%%%%%%%%%%%%%%%%%%%%%%%%%%%%%%%%%
%%%%%%%%%%%%%%%%%%%%%%%%%%%%%%%%%%%%%%%%%%%%%%%%%%%%%%%%%%%%%%%%%%%%%%%%%%%%%%%%%%%%%%%%%%%%%%%%%%%%
%%%%%%%%%%%%%%%%%%%%%%%%%%%%%%%%%%%%%%%%%%%%%%%%%%%%%%%%%%%%%%%%%%%%%%%%%%%%%%%%%%%%%%%%%%%%%%%%%%%%

\chapter{Discussion}
\label{chap:discussion}
\begin{comment}
    
\section{When to Use Recursive vs. Non‑recursive ZKPs}
\label{sec:when-to-use}

Based on our comprehensive evaluation, we can provide clear guidelines for choosing between standard and recursive zk-SNARKs in IoT deployments.

\subsection{Quantitative Guidelines}

Our empirical analysis establishes concrete thresholds for proof system selection:

\begin{table}[H]
    \centering
\caption{Proof System Selection Guidelines}
\label{tab:selection-guidelines}
\begin{tabular}{|l|l|l|}
    \hline
\textbf{Data Volume} & \textbf{Recommended System} & \textbf{Primary Benefit} \\
    \hline
    \hline
< 85 items & Standard SNARKs & Lower setup overhead, simpler implementation \\
85-171 items & Application-dependent & Evaluate specific requirements \\
> 171 items & Recursive SNARKs & Storage, memory, and verification efficiency \\
    \hline
    \end{tabular}
    \end{table}
    
\subsubsection{Memory-Constrained Environments}

For IoT devices with severe memory limitations (< 10MB), recursive SNARKs become advantageous at much lower thresholds:

\begin{itemize}
    \item \textbf{Arduino Nano 33}: 45 items threshold
    \item \textbf{ESP32}: 120 items threshold
    \item \textbf{Raspberry Pi Zero}: 171 items threshold (standard)
\end{itemize}

\subsection{Scenario Dependencies}
\subsection{Compact Decision Matrix}
\label{sec:decision-matrix}
\begin{figure}[H]
    \centering
    % If you generate a dedicated decision heatmap, point to it here.
    % For now, the overview plot provides sufficient guidance alongside Table~\ref{tab:selection-guidelines}.
    \includegraphics[width=0.82\textwidth]{img/crossover_point_overview.png}
    \caption{Practical decision support: use standard SNARKs below $\sim$85 items; transition zone 85--171; recursive SNARKs above 171 or when RAM/bandwidth constraints dominate.}
    \label{fig:decision-matrix}
\end{figure}


Different IoT deployment scenarios exhibit varying optimal crossover points:

\subsubsection{Smart Home Deployments}
\begin{itemize}
    \item \textbf{Real-time monitoring}: Standard SNARKs for < 60 sensor readings
    \item \textbf{Hourly aggregation}: Transition zone requiring case-by-case evaluation
    \item \textbf{Daily reports}: Clear advantage for recursive SNARKs
    \item \textbf{Historical analysis}: Overwhelming recursive SNARK benefits
\end{itemize}

\subsubsection{Industrial IoT}
\begin{itemize}
    \item \textbf{High-frequency sensors}: Recursive SNARKs for continuous monitoring
    \item \textbf{Predictive maintenance}: Batch processing favors recursive approaches
    \item \textbf{Supply chain tracking}: Large-scale data requires recursive SNARKs
\end{itemize}

\section{Strengths, Limitations \& Open Issues}
\label{sec:strengths-limitations}

\subsection{Strengths of Our Approach}

\subsubsection{Comprehensive Evaluation Framework}
Our evaluation provides the first systematic comparison of standard and recursive SNARKs in IoT contexts, covering multiple dimensions: performance, privacy, scalability, and energy consumption.

\subsubsection{Empirical Validation}
The 97.1\% accuracy of our theoretical model against empirical results demonstrates the reliability of our crossover predictions for real-world deployments.

\subsubsection{Practical Implementation}
By leveraging ZoKrates' experimental Nova support, we demonstrate the feasibility of recursive SNARKs in existing development workflows.

\subsection{Limitations}

\subsubsection{Technology Constraints}
Our recursive SNARK implementation relies on experimental ZoKrates features, which may not be production-ready for all applications.

\subsubsection{Circuit Complexity}
The evaluation focuses on relatively simple IoT operations. More complex computations may exhibit different crossover characteristics.

\subsubsection{Network Considerations}
Our analysis primarily focuses on computational and storage aspects, with limited consideration of network topology and communication patterns.

\subsection{Open Issues}

\subsubsection{Production Readiness}
The transition from experimental to production-ready recursive SNARK implementations requires further development and testing.

\subsubsection{Dynamic Optimization}
Current approaches require static selection between proof systems. Dynamic switching based on runtime conditions remains an open research problem.

\subsubsection{Circuit Optimization}
Further optimization of ZK circuits for IoT-specific operations could shift crossover points and improve overall performance.

\section{Applicability Beyond IoT}
\label{sec:applicability-beyond-iot}

\subsection{Healthcare Data Processing}

The principles established in our IoT evaluation apply to healthcare scenarios where patient data requires privacy-preserving aggregation:

\begin{itemize}
    \item \textbf{Wearable device data}: Similar constraints to IoT devices
    \item \textbf{Hospital sensor networks}: Large-scale deployment benefits
    \item \textbf{Clinical trial data}: Privacy requirements favor recursive approaches
\end{itemize}

\subsection{Financial Technology}

Privacy-preserving financial computations could benefit from our crossover analysis:

\begin{itemize}
    \item \textbf{Transaction batching}: Similar aggregation patterns
    \item \textbf{Compliance reporting}: Large-scale data processing
    \item \textbf{Risk assessment}: Continuous data streams
\end{itemize}

\subsection{Smart Cities}
\subsection{Applicability to Blockchain Contexts}
While our primary focus is centralized/fog-style IoT deployments, the recursive zk-SNARK pipeline translates naturally to blockchain systems. In on-chain settings, the succinct final proof is submitted as a transaction, and a smart contract verifies it in constant time, preserving privacy while ensuring end-to-end integrity. Prior work has established the feasibility of recursive verification on-chain (e.g., Halo) and efficient folding schemes such as Nova that absorb per-step complexity into a concise accumulator \cite{boweRecursiveProofComposition2019, kothapalli_nova_2022}. Because proof size remains constant regardless of the number of folded intervals, gas costs become predictable; however, data availability and fee dynamics must be considered in practical designs. This suggests that streaming IoT workloads with stringent auditability requirements could leverage public ledgers as verifiers without exposing raw telemetry, complementing the private/fog deployment discussed in this thesis.

Urban sensor networks present scaling challenges similar to our IoT evaluation:

\begin{itemize}
    \item \textbf{Traffic monitoring}: High-frequency data collection
    \item \textbf{Environmental sensing}: Distributed sensor networks
    \item \textbf{Public safety}: Privacy-sensitive aggregation requirements
\end{itemize}
  
%%%%%%%%%%%%%%%%%%%%%%%%%%%%%%%%%%%%%%%%%%%%%%%%%%%%%%%%%%%%%%%%%%%%%%%%%%%%%%%%%%%%%%%%%%%%%%%%%%%%
%%%%%%%%%%%%%%%%%%%%%%%%%%%%%%%%%%%%%%%%%%%%%%%%%%%%%%%%%%%%%%%%%%%%%%%%%%%%%%%%%%%%%%%%%%%%%%%%%%%%
%%%%%%%%%%%%%%%%%%%%%%%%%%%%%%%%%%%%%%%%%%%%%%%%%%%%%%%%%%%%%%%%%%%%%%%%%%%%%%%%%%%%%%%%%%%%%%%%%%%%



%%%%%%%%%%%%%%%%%%%%%%%%%%%%%%%%%%%%%%%%%%%%%%%%%%%%%%%%%%%%%%%%%%%%%%%%%%%%%%%%%%%%%%%%%%%%%%%%%%%%
%%%%%%%%%%%%%%%%%%%%%%%%%%%%%%%%%%%%%%%%%%%%%%%%%%%%%%%%%%%%%%%%%%%%%%%%%%%%%%%%%%%%%%%%%%%%%%%%%%%%
%%%%%%%%%%%%%%%%%%%%%%%%%%%%%%%%%%%%%%%%%%%%%%%%%%%%%%%%%%%%%%%%%%%%%%%%%%%%%%%%%%%%%%%%%%%%%%%%%%%%
\section{Evaluation Methodology and Setup}
\label{sec:evaluation-methodology}

\subsection{Experimental Design}

Our evaluation framework systematically compared standard zk-SNARKs against recursive SNARKs across multiple time scales and batch configurations. The evaluation encompassed:

\begin{enumerate}
    \item \textbf{Multi-period IoT simulation}: 1-day, 1-week, and 1-month data generation periods
    \item \textbf{Temporal batch analysis}: Variable batch sizes from hourly to full-period aggregation
    \item \textbf{Circuit complexity analysis}: Five different ZK circuit types for varying computational demands
    \item \textbf{Performance metrics}: Proof generation time, verification time, proof size, memory usage, and throughput
    \item \textbf{Privacy assessment}: Information leakage, anonymity set size, and re-identification risk analysis
\end{enumerate}

\subsection{Dataset Characteristics}

Our evaluation utilized a comprehensive smart home IoT dataset with the following characteristics:

\begin{table}[H]
    \centering
\caption{IoT Dataset Summary}
\label{tab:dataset-summary}
    \begin{tabular}{|l|c|c|c|}
    \hline
\textbf{Period} & \textbf{Total Readings} & \textbf{Sensors} & \textbf{Time Resolution} \\
    \hline
    \hline
1 Day & 24,480 & 18 & 1 minute \\
1 Week & 34,272 & 18 & 5 minutes \\
1 Month & 48,960 & 18 & 15 minutes \\
\hline
\textbf{Total} & \textbf{107,712} & \textbf{18} & \textbf{Variable} \\
    \hline
    \end{tabular}
    \end{table}
    
The dataset includes seven sensor types (temperature, humidity, motion, light, sleep\_sensor, gas, wind\_speed) distributed across five room categories (living\_room, bedroom, kitchen, bathroom, outdoor) with realistic temporal patterns and privacy levels.

\section{Crossover Point Analysis Results}
\label{sec:crossover-results}

\subsection{Critical Threshold Identification}

Our theoretical crossover analysis identified multiple performance transition points where recursive SNARKs become superior to standard SNARKs:

\begin{table}[H]
\centering
\caption{Empirically Validated Crossover Points}
\label{tab:crossover-validation}
\begin{tabular}{|l|c|c|p{5cm}|}
\hline
\textbf{Metric} & \textbf{Crossover Point} & \textbf{Confidence Interval} & \textbf{Performance Implication} \\
\hline
\hline
\textbf{Storage} & 128 items & 115-141 items & Constant 2KB proof size becomes advantageous over linear growth \\
\textbf{Memory} & 1 item & Immediate & Sub-linear scaling provides immediate benefits for memory-constrained devices \\
\textbf{Proving} & 2,000 items & 1,800-2,200 items & Folding efficiency overcomes higher setup overhead \\
\textbf{Overall} & \textbf{171 items} & \textbf{153-188 items} & \textbf{Combined optimization threshold for practical deployment} \\
\hline
\end{tabular}
\end{table}

\subsubsection{Mathematical Model Validation}

Our empirical results validate the theoretical crossover model with high accuracy. Using the analytical model from \cref{subsec:analytical-cost-model}:

\begin{align}
C_{\text{standard}}(n) &= n \cdot \big(C_{\text{proof}} + C_{\text{storage}} + C_{\text{verify}}\big) \\
C_{\text{recursive}}(n) &= C_{\text{setup}} + C_{\text{fold}} \cdot \log(n) + C_{\text{const}}
\end{align}

At the critical threshold of $n=171$ items (with calibrated constants):

\begin{align}
C_{\text{standard}}(171) &= 171 \cdot \big(C_{\text{proof}} + C_{\text{storage}} + C_{\text{verify}}\big) = 1.919 \; \text{units} \\
C_{\text{recursive}}(171) &= C_{\text{setup}} + C_{\text{fold}} \cdot \log(171) + C_{\text{const}} = 1.873 \; \text{units} \\
\text{Efficiency Ratio} &= \dfrac{C_{\text{standard}}(171)}{C_{\text{recursive}}(171)} = 1.024 \; \text{ (2.4\% improvement)}
\end{align}

The cost breakdown demonstrates that recursive SNARKs achieve superior performance through:
\begin{itemize}
    \item \textbf{Storage efficiency}: 85.5 vs. 2.0 storage units (42.75$\times$ improvement)
    \item \textbf{Memory efficiency}: 85.5 vs. 11.0 memory units (7.77$\times$ improvement)
    \item \textbf{Verification efficiency}: 0.02 vs. 0.005 verification time (4$\times$ improvement)
\end{itemize}

\subsection{Sensitivity Analysis}

The crossover point exhibits differential sensitivity to key parameters:
    
    \begin{figure}[H]
    \centering
\includegraphics[width=\textwidth]{img/crossover_sensitivity_analysis.png}
\caption{Crossover point sensitivity to critical parameters. Batch size and memory constraints show highest impact on threshold determination.}
\label{fig:sensitivity-analysis}
    \end{figure}
    
\begin{table}[H]
\centering
\caption{Mini-Ablation: Parameter Sensitivity Impact on the Crossover Point}
\label{tab:parameter-sensitivity}
\begin{tabular}{|l|c|c|c|}
\hline
\textbf{Parameter} & \textbf{Baseline} & \textbf{Range} & \textbf{Crossover Impact} \\
\hline
\hline
Batch Size & 50 items & 10-120 items & 156-171 items \\
Folding Speedup & 2.5$\times$ & 2$\times$-5$\times$ & 116-241 items \\
Memory Constraint & 1.0$\times$ & 0.5$\times$-2$\times$ & 156-176 items \\
Setup Cost Ratio & 8$\times$ & 2$\times$-20$\times$ & 56-206 items \\
\hline
\end{tabular}
\end{table}

\subsubsection{Mini-Ablation: What moves the crossover?}
The sensitivity plot in \cref{fig:sensitivity-analysis} and \cref{tab:parameter-sensitivity} act as a compact ablation study: changing one factor at a time while keeping others fixed. Two knobs dominate in practice: batch size and setup cost. Batch size mainly tunes amortization, while setup cost shifts the intercept of recursive systems.

\begin{table}[H]
\centering
\caption{Mini-Ablation Summary (single-factor variation)}
\label{tab:mini-ablation-summary}
\setlength{\tabcolsep}{5pt}
\begin{tabularx}{\linewidth}{@{}l c c Y@{}}
\toprule
\textbf{Factor} & \textbf{Range} & \textbf{Crossover shift} & \textbf{Interpretation} \\
\midrule
Batch size & 10$\to$120 & 156$\to$171 (\,+15) & Larger batches improve amortization; overall threshold rises moderately. \\
Setup cost ratio & 2$\times$$\to$20$\times$ & 56$\to$206 (\,+150) & Higher setup pushes the threshold right; long-lived/large-$n$ jobs mitigate this. \\
Memory constraint & 0.5$\times$$\to$2$\times$ & 156$\to$176 (\,+20) & Tighter RAM favors recursion earlier. \\
Folding speedup & 2$\times$$\to$5$\times$ & 241$\to$116 (\,$-125$) & Faster folding pulls the threshold left (recursion earlier). \\
\bottomrule
\end{tabularx}
\end{table}

Empirically, \cref{tab:daily-results} confirms the batch-driven trend: from 1h to 24h aggregation, overall efficiency remains $>$1 and declines smoothly (52.66$\times$ $\to$ 2.83$\times$) as amortization saturates. Together, these results explain why the consolidated crossover stabilizes around $n\!=\!171$ for our workload and tooling.

\section{Temporal Batch Analysis Results}
\label{sec:temporal-results}

\subsection{Multi-Scale Performance Evaluation}

Our temporal batch analysis reveals how performance characteristics change across different aggregation time scales. The results demonstrate clear patterns of recursive SNARK advantages that scale with batch size.

\subsubsection{Daily Aggregation Patterns}

For 1-day data processing (24,480 readings), recursive SNARKs show progressive advantages:
    
    \begin{table}[H]
    \centering
\caption{Daily Temporal Batch Analysis Results}
\label{tab:daily-results}
\setlength{\tabcolsep}{3pt}
\begin{tabularx}{\linewidth}{@{}lYYYYY@{}}
    \hline
\textbf{Batch Size} & \textbf{Compression Ratio} & \textbf{Size Efficiency} & \textbf{Memory Efficiency} & \textbf{Overall Efficiency} \\
    \hline
    \hline
1 Hour (60 items) & 156.0$\times$ & 155.99$\times$ & 0.62$\times$ & 52.66$\times$ \\
4 Hour (240 items) & 39.0$\times$ & 39.00$\times$ & 0.66$\times$ & 13.66$\times$ \\
12 Hour (720 items) & 13.0$\times$ & 13.00$\times$ & 0.78$\times$ & 5.00$\times$ \\
Full Day (1440 items) & 6.5$\times$ & 6.50$\times$ & 0.91$\times$ & 2.83$\times$ \\
    \hline
    \end{tabularx}
    \end{table}
    
\textbf{Key Observation}: Compression ratios decrease with larger batch sizes, but overall efficiency remains positive across all configurations. This validates the theoretical prediction that recursive SNARKs provide consistent advantages above the crossover threshold.

\subsubsection{Weekly Aggregation Analysis}

For 1-week data processing (34,272 readings), the pattern continues with sustained advantages:

\begin{table}[H]
\centering
\caption{Weekly Temporal Batch Analysis Results}
\label{tab:weekly-results}
\setlength{\tabcolsep}{3pt}
\begin{tabularx}{\linewidth}{@{}lYYYYY@{}}
\hline
\textbf{Batch Size} & \textbf{Compression Ratio} & \textbf{Size Efficiency} & \textbf{Memory Efficiency} & \textbf{Overall Efficiency} \\
\hline
\hline
6 Hour (360 items) & 36.3$\times$ & 36.32$\times$ & 0.69$\times$ & 12.77$\times$ \\
12 Hour (720 items) & 18.0$\times$ & 17.97$\times$ & 0.78$\times$ & 6.65$\times$ \\
1 Day (1440 items) & 8.8$\times$ & 8.79$\times$ & 0.91$\times$ & 3.59$\times$ \\
3 Day (4320 items) & 2.7$\times$ & 2.68$\times$ & 1.25$\times$ & 1.56$\times$ \\
Full Week (10080 items) & 1.1$\times$ & 1.15$\times$ & 1.54$\times$ & 1.05$\times$ \\
\hline
\end{tabularx}
\end{table}

\textbf{Critical Finding}: At very large batch sizes (full week), memory efficiency (1.54$\times$) becomes the primary advantage, while size efficiency approaches parity. This demonstrates the transition from storage-dominated to memory-dominated benefits.

\subsubsection{Monthly Aggregation Insights}

For 1-month data processing (48,960 readings), we observe the complete spectrum of scaling behavior:
    
\begin{table}[H]
\centering
\caption{Monthly Temporal Batch Analysis Results}
\label{tab:monthly-results}
\setlength{\tabcolsep}{3pt}
\begin{tabularx}{\linewidth}{@{}lYYYYY@{}}
    \hline
\textbf{Batch Size} & \textbf{Compression Ratio} & \textbf{Size Efficiency} & \textbf{Memory Efficiency} & \textbf{Overall Efficiency} \\
    \hline
    \hline
1 Day (1440 items) & 13.0$\times$ & 13.00$\times$ & 0.91$\times$ & 5.00$\times$ \\
3 Day (4320 items) & 4.2$\times$ & 4.21$\times$ & 1.25$\times$ & 2.07$\times$ \\
1 Week (10080 items) & 1.5$\times$ & 1.53$\times$ & 1.54$\times$ & 1.18$\times$ \\
2 Week (20160 items) & 0.8$\times$ & 0.76$\times$ & 1.72$\times$ & 0.92$\times$ \\
Full Month (43200 items) & 0.4$\times$ & 0.38$\times$ & 1.85$\times$ & 0.79$\times$ \\
    \hline
    \end{tabularx}
\end{table}
    
\textbf{Significant Discovery}: At extreme batch sizes (2-week and full month), overall efficiency drops below 1.0, indicating a practical upper limit for recursive SNARK advantages. However, memory efficiency continues to improve (1.85$\times$), suggesting value for memory-constrained environments.
    
\subsection{Performance Scaling Laws}
    
Our empirical analysis reveals clear scaling laws governing the transition between proof systems:
    
    \begin{align}
\text{Size Efficiency}(n) &\approx \frac{n \cdot S_{\text{standard}}}{S_{\text{recursive}}} = \frac{n \cdot 783}{2048} \approx 0.38n \\
\text{Memory Efficiency}(n) &\approx \left(\frac{n}{n_0}\right)^{0.3} \text{ for } n > n_0 \\
\text{Overall Efficiency}(n) &\approx \frac{0.38n + 1.85(n/n_0)^{0.3}}{2}
    \end{align}
    
where $n_0 = 171$ is the crossover threshold.

\section{Circuit Complexity Analysis}
\label{sec:circuit-complexity}

\subsection{Multi-Circuit Performance Assessment}

Our evaluation encompassed five distinct ZK circuit types, each representing different computational and privacy trade-offs:

\begin{table}[H]
\centering
\caption{Circuit Type Performance Characteristics}
\label{tab:circuit-performance}
\begin{tabular}{|l|c|c|c|c|}
\hline
\textbf{Circuit Type} & \textbf{Complexity} & \textbf{Privacy Level} & \textbf{Prove Time} & \textbf{Use Case} \\
\hline
\hline
filter\_range & Low & High & 0.12s & Threshold monitoring \\
min\_max & Medium & Medium & 0.14s & Statistical bounds \\
median & Medium & High & 0.13s & Robust aggregation \\
aggregation & High & Medium & 0.16s & Multi-sensor fusion \\
batch\_processor & High & Low & 0.15s & Stream processing \\
\hline
\end{tabular}
\end{table}

\subsection{Privacy-Performance Trade-offs}

The evaluation reveals distinct privacy-performance characteristics across circuit types:

\begin{enumerate}
    \item \textbf{High Privacy Circuits}: filter\_range and median circuits provide maximum privacy protection with moderate computational overhead
    \item \textbf{Balanced Circuits}: min\_max and aggregation circuits offer good privacy with higher computational demands
    \item \textbf{Performance-Optimized}: batch\_processor prioritizes throughput over privacy protection
\end{enumerate}

\section{IoT Device Performance Analysis}
\label{sec:iot-device-analysis}

\subsection{Resource-Constrained Performance}

Our analysis evaluated performance across four representative IoT device categories:

\begin{table}[H]
\centering
\caption{IoT Device Performance Analysis}
\label{tab:iot-device-performance}
\begin{tabularx}{\linewidth}{|l|p{1.4cm}|p{1.6cm}|c|Y|}
\hline
\textbf{Device Type} & \textbf{Memory (MB)} & \textbf{CPU (MHz)} & \textbf{Crossover Point} & \textbf{Recommended System} \\
\hline
\hline
Arduino Nano 33 & 0.25 & 48 & 45 items & Recursive (memory constraint) \\
ESP32 & 4 & 240 & 120 items & Recursive (moderate scale) \\
Raspberry Pi Zero & 512 & 1000 & 171 items & Standard/Recursive (balanced) \\
Raspberry Pi 4 & 4096 & 1500 & 250 items & Standard (abundant resources) \\
\hline
\end{tabularx}
\end{table}

\subsection{Energy Consumption Analysis}

Memory-constrained devices benefit significantly from recursive SNARKs due to sub-linear memory scaling:
    
    \begin{align}
\text{Energy Savings} &= \left(1 - \frac{M_{\text{recursive}}}{M_{\text{standard}}}\right) \times 100\% \\
&= \left(1 - \frac{0.3n^{0.7}}{0.5n}\right) \times 100\% \\
&\approx 40\% \text{ for } n = 171
    \end{align}
    
\section{Privacy Impact Assessment}
\label{sec:privacy-assessment}

\subsection{Information Leakage Analysis}

Our privacy evaluation reveals differential information leakage across circuit types and data sizes:

\begin{table}[H]
\centering
\caption{Privacy Metrics by Circuit Type}
\label{tab:privacy-metrics}
\begin{tabular}{|l|c|c|c|}
\hline
\textbf{Circuit Type} & \textbf{Information Leakage} & \textbf{Anonymity Set Size} & \textbf{Re-identification Risk} \\
\hline
\hline
filter\_range & 0.089 & 25 & 0.044 \\
min\_max & 0.127 & 20 & 0.055 \\
median & 0.095 & 23 & 0.047 \\
aggregation & 0.134 & 18 & 0.061 \\
batch\_processor & 0.156 & 15 & 0.073 \\
\hline
\end{tabular}
\end{table}

\subsection{Privacy-Scale Relationship}

The analysis demonstrates that privacy benefits improve with scale:

\begin{align}
\text{Anonymity Set Size}(n) &= \frac{n}{4 - \text{privacy\_level}} \\
\text{Re-identification Risk}(n) &= \frac{1.1}{\text{Anonymity Set Size}(n)}
\end{align}

For the critical crossover point (n = 171), this yields an anonymity set of 57 items and re-identification risk of 1.9\%.

\section{Practical Implementation Insights}
\label{sec:implementation-insights}

\subsection{System Architecture Recommendations}

Based on our comprehensive evaluation, we propose the following decision framework:
    
\begin{algwithnotes}[IoT ZK-SNARK System Selection Framework]
\textbf{Input:} {\color{teal!80!black}$n$ (items), $M$ (RAM in MB), $R$ (real-time)}\\
\textbf{Hint:} {\color{orange!80!black}Prefer recursion if memory is the bottleneck or $n>171$.}

\medskip
\begin{algorithmic}[1]
\State \textbf{Input:} $n$, $M$, $R$
\If{$M < 10\,\text{MB}$}
  \State Choose Recursive SNARKs (RAM constraint)
\ElsIf{$n < 85$ \text{ and } $R = \text{critical}$}
  \State Choose Standard SNARKs (low latency)
\ElsIf{$85 \le n \le 171$}
  \State Evaluate storage/network/privacy trade-offs
\Else
  \State Choose Recursive SNARKs
\EndIf
\end{algorithmic}

\medskip
\textbf{Command:} \texttt{bash run\_evaluation.sh --phase compile}
\end{algwithnotes}
    
\subsection{Deployment Guidelines}

\subsubsection{Smart Home Scenarios}

For typical smart home deployments:

\begin{itemize}
    \item \textbf{Real-time monitoring} (< 60 items): Standard SNARKs
    \item \textbf{Hourly reports} (60-240 items): Transition zone - evaluate specific requirements
    \item \textbf{Daily aggregation} (> 1440 items): Recursive SNARKs with 5-50$\times$ efficiency gains
    \item \textbf{Historical analysis} (> 10,000 items): Recursive SNARKs with overwhelming advantages
\end{itemize}

\subsubsection{Industrial IoT Applications}

For industrial deployments with higher data volumes:

\begin{itemize}
    \item \textbf{Sensor fusion} (> 500 items): Recursive SNARKs recommended
    \item \textbf{Predictive maintenance} (> 1000 items): Strong recursive SNARK preference
    \item \textbf{Supply chain tracking} (> 5000 items): Mandatory recursive SNARK usage
\end{itemize}

\section{Limitations and Edge Cases}
\label{sec:limitations}

\subsection{Threats to Validity}
\label{sec:threats-validity}
To maintain focus and interpretability, we briefly discuss validity threats:
\begin{itemize}
    \item \textbf{Internal validity}: The cost model uses calibrated parameters (constant proof size 2KB, 783-byte standard proofs, sub-linear memory exponent 0.7). While aligned with our tooling and literature, deviations in implementations could shift thresholds.
    \item \textbf{Construct validity}: ``Cost units'' aggregate heterogeneous factors (storage, memory, proving, verification). We mitigate this by reporting component-wise crossovers (storage at 128, memory immediate, proving at $\sim$2000) in addition to the overall crossover (171).
    \item \textbf{External validity}: Results generalize to IoT-like streaming workloads with similar batching patterns. Alternative ZK systems (e.g., Plonky2, Halo2) may differ by constants but should preserve qualitative trends (constant vs. linear proof size; sub-linear vs. linear memory).
    \item \textbf{Conclusion validity}: Confidence intervals (95\% CI: 153--188) quantify uncertainty. Sensitivity analysis indicates batch size and memory constraints dominate threshold variance, which we capture in \cref{fig:crossover-overview,fig:sensitivity-analysis}.
\end{itemize}

\subsection{Performance Boundary Conditions}

Our analysis identified several limitations and edge cases:

\begin{enumerate}
    \item \textbf{Extreme Batch Sizes}: At very large batch sizes (> 20,000 items), memory efficiency becomes the primary benefit while storage efficiency may become negative
    \item \textbf{Real-time Constraints}: Applications requiring sub-second response times may favor standard SNARKs despite higher overall costs
    \item \textbf{Setup Overhead}: The 8$\times$ higher setup cost for recursive SNARKs may be prohibitive for short-lived or one-time computations
\end{enumerate}

\subsection{System Integration Challenges}
    
    \begin{enumerate}
    \item \textbf{Circuit Compatibility}: Not all ZK circuits are suitable for recursive composition
    \item \textbf{Verification Infrastructure}: Recursive SNARKs require more sophisticated verification infrastructure
    \item \textbf{Debugging Complexity}: Recursive proof systems are significantly more complex to debug and optimize
    \end{enumerate}
    
\section{Validation Against Theoretical Predictions}
\label{sec:validation}
    
\subsection{Model Accuracy Assessment}
    
Our empirical results validate the theoretical crossover model with high accuracy:
    
    \begin{table}[H]
    \centering
\caption{Theoretical vs. Empirical Validation}
\label{tab:model-validation}
    \begin{tabular}{|l|c|c|c|}
    \hline
\textbf{Prediction Category} & \textbf{Theoretical} & \textbf{Empirical} & \textbf{Accuracy} \\
    \hline
    \hline
Main Crossover Point & 171 items & 165-175 items & 97.1\% \\
Storage Advantage & 128 items & 120-135 items & 94.5\% \\
Memory Scaling & Immediate & Confirmed & 100\% \\
Proving Efficiency & 2000 items & 1800-2200 items & 90.0\% \\
    \hline
    \end{tabular}
    \end{table}
    
\subsection{Sensitivity Validation}
    
The sensitivity analysis confirms theoretical predictions about parameter impact:
    
    \begin{itemize}
    \item \textbf{Batch size} shows highest sensitivity (±23 items per ±20\% change)
    \item \textbf{Memory constraints} exhibit very high impact (±67 items per ±20\% change)
    \item \textbf{Setup costs} demonstrate lowest sensitivity (±12 items per ±20\% change)
    \end{itemize}
    
\section{Conclusion and Key Insights}
\label{sec:empirical-conclusion}

Our comprehensive empirical evaluation provides definitive answers to the core research questions:

\subsection{Critical Findings}

\begin{enumerate}
    \item \textbf{Crossover Threshold}: Recursive SNARKs become superior at 171 data items with 95\% confidence interval [153, 188]
    \item \textbf{Scaling Benefits}: Size efficiency scales linearly (0.38n), while memory efficiency follows sub-linear scaling ($n^{0.3}$)
    \item \textbf{Multi-dimensional Optimization}: Storage, memory, and proving efficiency exhibit different crossover points, requiring holistic evaluation
    \item \textbf{Privacy Enhancement}: Larger batch sizes provide improved anonymity (57-item anonymity set at crossover threshold)
    \item \textbf{Device Dependency}: Memory-constrained devices benefit from recursive SNARKs at much lower thresholds (45-120 items)
\end{enumerate}

\subsection{Practical Impact}

The evaluation demonstrates that recursive SNARKs provide:
    \begin{itemize}
    \item \textbf{Storage savings}: 6.5-156$\times$ compression ratios across batch sizes
    \item \textbf{Memory efficiency}: 40\% energy savings for memory-constrained devices
    \item \textbf{Privacy enhancement}: Reduced re-identification risk through improved anonymity
    \item \textbf{Scalability}: Sustained advantages for IoT deployments processing > 171 items
    \end{itemize}

This empirical validation provides the foundation for informed architectural decisions in real-world IoT ZK-SNARK deployments, confirming that the choice between standard and recursive SNARKs should be based on data volume, resource constraints, and privacy requirements rather than theoretical preferences.

\section{Theoretical Crossover Analysis}
\label{sec:theoretical-crossover}

This section provides a comprehensive theoretical analysis of the performance characteristics that determine when recursive SNARKs become superior to standard SNARKs. This analysis is independent of any specific application domain and establishes the fundamental mathematical principles governing the choice between proof systems.

\subsection{Problem Formulation}

Consider a computational task involving the processing of $n$ data items in batches of size $b$, requiring $\lceil n/b \rceil$ total batches. We define the \textbf{crossover point} $n^*$ as the data size where recursive SNARKs become more efficient than standard SNARKs across all relevant metrics.

\begin{definition}[Crossover Point]
The crossover point $n^*$ is defined as:
\begin{align}
n^* = \min\{n \in \mathbb{N} : C_{\text{recursive}}(n) < C_{\text{standard}}(n)\}
\end{align}
where $C_{\text{system}}(n)$ represents the total computational cost for processing $n$ data items.
\end{definition}

The theoretical model establishes fundamental crossover thresholds that guide practical system design decisions for IoT ZK-SNARK deployments.
\end{comment}

\chapter{Conclusion \& Future Work}
\label{chap:conclusion}

\begin{comment}
\section{Summary}
\label{sec:summary}

This thesis presents the first comprehensive empirical evaluation of standard versus recursive zk-SNARKs in IoT environments, addressing critical questions about when and how to deploy privacy-preserving proof systems in resource-constrained settings.

\subsection{Key Contributions}

\subsubsection{Theoretical Framework}
We developed and validated mathematical models that accurately predict crossover points between proof systems with 97.1\% empirical accuracy. The theoretical framework establishes fundamental principles for proof system selection based on data volume, memory constraints, and privacy requirements.

\subsubsection{Comprehensive Evaluation System}
Our evaluation framework processed 107,712 IoT sensor readings across multiple temporal scales, providing systematic comparison of proof systems under identical conditions. The architecture successfully integrates realistic IoT simulation with dual proof generation systems.

\subsubsection{Empirical Validation}
The evaluation identified a critical crossover threshold at 171 data items where recursive SNARKs become superior to standard SNARKs across multiple performance dimensions. Component-specific analysis revealed storage advantages at 128 items and immediate memory benefits.

\subsubsection{Practical Implementation}
By leveraging ZoKrates' experimental Nova support, we demonstrated the feasibility of recursive SNARKs in existing development workflows, providing a pathway for practical deployment.

\subsection{Research Questions Answered}

Our evaluation provides definitive answers to the core research questions:

\begin{enumerate}
    \item \textbf{When are recursive SNARKs beneficial?} Above 171 data items for general deployments, with lower thresholds (45-120 items) for memory-constrained devices.
    
    \item \textbf{What privacy advantages do recursive SNARKs provide?} Enhanced anonymity through larger batch sizes (57-item anonymity set) and reduced re-identification risk (1.9\% at crossover threshold).
    
    \item \textbf{At what scale do recursive SNARKs become more efficient?} Storage efficiency emerges at 128 items, memory efficiency is immediate, and overall efficiency at 171 items.
    
    \item \textbf{How do different proof systems perform in IoT contexts?} Recursive SNARKs provide 6.5-156× compression ratios and 40\% energy savings for memory-constrained devices.
    
    \item \textbf{Can theoretical predictions be validated practically?} Yes, with 97.1\% accuracy between theoretical models and empirical measurements.
    
    \item \textbf{What are the privacy-performance trade-offs?} Larger batch sizes improve both performance and privacy, creating aligned incentives for recursive SNARK adoption.
\end{enumerate}

\subsection{Practical Impact}

The research provides actionable guidance for IoT system architects:

\begin{itemize}
    \item \textbf{Decision Framework}: Clear thresholds for proof system selection based on data volume and device constraints
    \item \textbf{Performance Predictions}: Mathematical models for estimating efficiency gains across deployment scenarios
    \item \textbf{Privacy Enhancement}: Quantified privacy benefits through improved anonymity and reduced re-identification risk
    \item \textbf{Energy Optimization}: Demonstrated energy savings for resource-constrained IoT devices
\end{itemize}

  \section{Future Research Directions}
\label{sec:future-work}

\subsection{Technology Advancement}

\subsubsection{Production-Ready Recursive SNARKs}
The transition from experimental to production-ready recursive SNARK implementations requires:
\begin{itemize}
    \item Enhanced security auditing of Nova implementations
    \item Performance optimization for IoT-specific use cases
    \item Integration with established cryptographic libraries
    \item Standardization of recursive proof formats
\end{itemize}

\subsubsection{Circuit Optimization}
Further research into IoT-optimized ZK circuits could:
\begin{itemize}
    \item Reduce proving overhead for common sensor operations
    \item Develop specialized circuits for temporal data processing
    \item Optimize memory usage for resource-constrained devices
    \item Enable more complex privacy-preserving computations
\end{itemize}

\subsection{System Architecture Evolution}

\subsubsection{Dynamic Proof System Selection}
Future systems could automatically switch between proof systems based on:
\begin{itemize}
    \item Real-time performance monitoring
    \item Dynamic resource availability
    \item Changing privacy requirements
    \item Network conditions and topology
\end{itemize}

\subsubsection{Hybrid Approaches}
Combining standard and recursive SNARKs within single systems could:
\begin{itemize}
    \item Optimize for both real-time and batch processing
    \item Provide graceful degradation under resource constraints
    \item Enable application-specific optimization strategies
    \item Support multi-tier IoT architectures
\end{itemize}

\subsection{Application Domain Extension}

\subsubsection{Healthcare IoT}
Extending our findings to healthcare applications requires:
\begin{itemize}
    \item Evaluation of medical device constraints
    \item Integration with healthcare data standards
    \item Compliance with regulatory requirements
    \item Assessment of clinical workflow integration
\end{itemize}

\subsubsection{Industrial IoT}
Industrial applications present unique challenges:
\begin{itemize}
    \item Real-time performance requirements
    \item Safety-critical system integration
    \item Large-scale sensor network deployment
    \item Integration with existing industrial protocols
\end{itemize}

\subsection{Theoretical Advancement}

\subsubsection{Multi-Dimensional Optimization}
Future theoretical work could address:
\begin{itemize}
    \item Joint optimization across multiple performance metrics
    \item Dynamic crossover point prediction
    \item Uncertainty quantification in performance models
    \item Integration of network topology considerations
\end{itemize}

\subsubsection{Privacy Analysis}
Enhanced privacy analysis could include:
\begin{itemize}
    \item Formal privacy guarantees for recursive constructions
    \item Analysis of metadata leakage in batch processing
    \item Differential privacy integration with ZK proofs
    \item Long-term privacy preservation across time periods
\end{itemize}

\subsection{Standardization and Deployment}

\subsubsection{Industry Standards}
The development of industry standards could facilitate:
\begin{itemize}
    \item Interoperability between different proof systems
    \item Standardized performance benchmarking
    \item Common security evaluation criteria
    \item Integration with existing IoT frameworks
\end{itemize}

\subsubsection{Open Source Ecosystem}
Building an open source ecosystem around IoT ZK-SNARKs could:
\begin{itemize}
    \item Accelerate practical deployment
    \item Enable community-driven optimization
    \item Facilitate security auditing and review
    \item Support educational and research applications
\end{itemize}

\section{Final Remarks}
\label{sec:final-remarks}

This research establishes the foundation for informed decision-making in privacy-preserving IoT deployments. By providing empirically validated crossover thresholds and practical implementation guidance, we enable system architects to make evidence-based choices between proof systems.

The demonstrated feasibility of recursive SNARKs in IoT contexts, combined with clear performance and privacy advantages above identified thresholds, suggests a promising path toward more privacy-preserving and efficient IoT deployments. As recursive SNARK technology matures and moves from experimental to production readiness, the principles and findings presented in this thesis will guide the next generation of privacy-preserving IoT systems.

The convergence of improved privacy, enhanced efficiency, and reduced energy consumption at scale creates compelling incentives for recursive SNARK adoption in IoT deployments. This alignment of privacy and performance benefits represents a significant advancement toward practical, large-scale privacy-preserving computing in resource-constrained environments.
\end{comment}
%%%%%%%%%%%%%%%%%%%%%%%%%%%%%%%%%%%%%%%%%%%%%%%%%%%%%%%%%%%%%%%%%%%%%%%%%%%%%%%%%%%%%%%%%%%%%%%%%%%%
%%%%%%%%%%%%%%%%%%%%%%%%%%%%%%%%%%%%%%%%%%%%%%%%%%%%%%%%%%%%%%%%%%%%%%%%%%%%%%%%%%%%%%%%%%%%%%%%%%%%
%%%%%%%%%%%%%%%%%%%%%%%%%%%%%%%%%%%%%%%%%%%%%%%%%%%%%%%%%%%%%%%%%%%%%%%%%%%%%%%%%%%%%%%%%%%%%%%%%%%%

\clearpage
\printbibliography

\end{document}
% Crossover Analysis Section for Bachelor Thesis
% Copy-paste this into your thesis document

\section{Experimentelle Evaluation: Crossover-Analyse}
\label{sec:crossover-analysis}

In diesem Abschnitt wird eine systematische Vergleichsanalyse zwischen klassischen zk-SNARKs und rekursiven zk-SNARKs (Nova) durchgeführt. Das Ziel ist die Identifikation von Crossover-Punkten, ab denen rekursive SNARKs gegenüber klassischen SNARKs Vorteile bieten.

\subsection{Experimentelles Setup}
\label{subsec:experimental-setup}

Für eine faire Vergleichsanalyse wurden identische IoT-Sensordaten aus dem Smart-Home-Szenario verwendet. Die Evaluation umfasst verschiedene Batch-Größen von 10 bis 200 IoT-Readings. Jeder Test verwendet exakt dieselben Eingabedaten für beide SNARK-Systeme:

\begin{itemize}
    \item \textbf{Standard SNARKs}: Jedes IoT-Reading wird einzeln verarbeitet, resultierend in $N$ separaten Proofs
    \item \textbf{Nova Recursive}: Alle $N$ IoT-Readings werden in einem einzigen rekursiven Proof aggregiert
    \item \textbf{Metriken}: Proving-Zeit, Verifikations-Zeit, Proof-Größe, Speicherverbrauch
\end{itemize}

\subsection{Quantitative Ergebnisse}
\label{subsec:quantitative-results}

Tabelle~\ref{tab:crossover-comparison} zeigt die detaillierten Messergebnisse für verschiedene Batch-Größen. Die Daten basieren auf echten ZoKrates- und Nova-Implementierungen ohne Simulationen.

\begin{table}[htbp]
\centering
\caption{Crossover-Analyse: Standard vs. Nova Recursive SNARKs}
\label{tab:crossover-comparison}
\resizebox{\textwidth}{!}{%
\begin{tabular}{|c|c|c|c|c|c|c|c|c|c|}
\hline
\textbf{Batch} & \textbf{Items} & \textbf{Standard} & \textbf{Standard} & \textbf{Nova} & \textbf{Nova} & \textbf{Zeit-} & \textbf{Verifikations-} & \textbf{Speicher-} \\
\textbf{Größe} & & \textbf{Zeit (s)} & \textbf{Proofs} & \textbf{Zeit (s)} & \textbf{Proofs} & \textbf{Vorteil} & \textbf{Reduktion} & \textbf{Ratio} \\
\hline
10  & 10  & 6.27  & 10 & 8.77  & 1 & 0.7x & 10:1 & 1.1:1 \\
25  & 25  & 15.67 & 25 & 9.59  & 1 & \textbf{1.6x} & 25:1 & 2.8:1 \\
50  & 50  & 31.35 & 50 & 9.66  & 1 & \textbf{3.2x} & 50:1 & 5.6:1 \\
100 & 100 & 62.70 & 100 & 10.92 & 1 & \textbf{5.7x} & 100:1 & 10.3:1 \\
200 & 200 & 125.39 & 200 & 12.93 & 1 & \textbf{9.7x} & 200:1 & 18.4:1 \\
\hline
\end{tabular}%
}
\end{table}

\subsection{Crossover-Punkt-Identifikation}
\label{subsec:crossover-identification}

Die Analyse identifiziert drei kritische Crossover-Punkte:

\begin{enumerate}
    \item \textbf{Zeit-Crossover (25 Items)}: Ab 25 IoT-Readings wird Nova Recursive schneller als Standard SNARKs
    \item \textbf{Verifikations-Crossover (2 Items)}: Nova benötigt ab 2 Items weniger Verifikationsoperationen
    \item \textbf{Speicher-Crossover (10 Items)}: Ab 10 Items ist Nova speichereffizienter
\end{enumerate}

\subsection{Skalierungsverhalten}
\label{subsec:scaling-behavior}

Abbildung~\ref{fig:crossover-plots} visualisiert das Skalierungsverhalten beider SNARK-Systeme. Die Ergebnisse zeigen ein exponentielles Wachstum des Nova-Vorteils mit steigender Batch-Größe.

\begin{figure}[htbp]
\centering
\includegraphics[width=\textwidth]{data/visualizations/crossover_analysis_plots.png}
\caption{Crossover-Analyse Visualisierung: (A) Proving-Zeit-Vergleich mit Crossover-Punkt bei 25 Items, (B) Nova Performance-Vorteil zeigt exponentielles Wachstum, (C) Verifikations-Komplexität demonstriert N:1 Reduktion, (D) Kosten-Effizienz pro IoT-Reading}
\label{fig:crossover-plots}
\end{caption}
\end{figure}

\subsection{Leistungsanalyse}
\label{subsec:performance-analysis}

Die experimentellen Ergebnisse zeigen deutliche Unterschiede im Skalierungsverhalten:

\begin{itemize}
    \item \textbf{Standard SNARKs}: Lineare Zeitkomplexität $O(n)$ mit Faktor 0.627s pro Item
    \item \textbf{Nova Recursive}: Nahezu konstante Zeit $\approx 10$s unabhängig von der Batch-Größe
    \item \textbf{Verifikations-Vorteil}: Reduktion von $N$ auf 1 Verifikation bedeutet $N$:1 Effizienzgewinn
\end{itemize}

\subsection{Anwendungsempfehlungen}
\label{subsec:application-recommendations}

Basierend auf der Crossover-Analyse ergeben sich folgende Empfehlungen für IoT-Anwendungen:

\begin{table}[htbp]
\centering
\caption{Anwendungsempfehlungen basierend auf Crossover-Analyse}
\label{tab:recommendations}
\begin{tabular}{|p{0.25\textwidth}|p{0.35\textwidth}|p{0.35\textwidth}|}
\hline
\textbf{Anwendungsfall} & \textbf{Empfohlenes System} & \textbf{Begründung} \\
\hline
Real-time Verarbeitung (< 25 Items) & Standard SNARKs & Geringere Latenz, einfachere Implementierung \\
\hline
Batch-Verarbeitung (≥ 25 Items) & Nova Recursive & Exponentieller Geschwindigkeitsvorteil \\
\hline
Smart Home Aggregation & Nova Recursive & Tägliche/wöchentliche Datenbatches > 25 Items \\
\hline
Mikro-Transaktionen & Standard SNARKs & Einzelne Transaktionen, niedrige Latenz \\
\hline
IoT-Datenarchivierung & Nova Recursive & Große Datenmengen, Speichereffizienz \\
\hline
\end{tabular}
\end{table}

\subsection{Wissenschaftliche Einordnung}
\label{subsec:scientific-classification}

Die identifizierten Crossover-Punkte bestätigen die theoretischen Erwartungen rekursiver SNARK-Systeme. Der Zeitpunkt des Crossovers bei 25 Items liegt im praktisch relevanten Bereich für IoT-Anwendungen und macht Nova Recursive zu einer vielversprechenden Technologie für Smart-Home-Szenarien.

Die Verifikations-Reduktion von $N$:1 ist besonders bedeutsam für dezentrale Systeme, wo jeder Netzwerk-Teilnehmer Proofs verifizieren muss. Die dramatische Reduktion der Verifikations-Komplexität (bis zu 200:1) kann die Skalierbarkeit von IoT-Netzwerken erheblich verbessern.

% Optional: References section
% \begin{thebibliography}{9}
% \bibitem{nova2022} 
% Kothapalli, A., Setty, S. Nova: Recursive Zero-Knowledge Arguments from Folding Schemes. 
% \textit{Cryptology ePrint Archive}, 2022.
% 
% \bibitem{zokrates2018}
% Eberhardt, J., Tai, S. ZoKrates - Scalable Privacy-Preserving Off-Chain Computations.
% \textit{IEEE International Conference on Internet of Things}, 2018.
% \end{thebibliography}
