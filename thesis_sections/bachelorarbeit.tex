\documentclass[english,bibtotoc,liststotoc,oneside,BCOR=5mm,DIV=12]{scrbook}
\recalctypearea

\usepackage[english]{babel}
\usepackage[T1]{fontenc}
\usepackage{lmodern}
\usepackage{microtype}
\usepackage{amsmath}
\usepackage{amssymb}
\usepackage{amsthm}
\usepackage{graphicx}
\usepackage{float}
\usepackage{url}
\usepackage{hyperref}
\usepackage[nameinlink]{cleveref}
\usepackage{pgfplots}
\usepackage{tikz}
\usetikzlibrary{positioning,fit,arrows.meta,backgrounds}
\usepackage{adjustbox} % für max width = \linewidth
\usepackage{booktabs}
\usepackage{siunitx}
\usepackage{csvsimple}   % moderne Version mit \csvautotabular
\usepackage{booktabs}
\usepackage{adjustbox}
\pgfplotsset{compat=1.18}
\usepackage{float}
\usepackage{listings, color}
\usepackage{subcaption}
\usetikzlibrary{positioning,arrows.meta}
\usepackage[automark]{scrlayer-scrpage}
\setlength{\marginparwidth}{2cm}
\usepackage{todonotes}
\usepackage{comment}
\usepackage{tikz}
\usetikzlibrary{shapes,arrows,positioning,fit,backgrounds,shadows,decorations.pathmorphing,decorations.pathreplacing}
\usetikzlibrary{calc,intersections,through,backgrounds,matrix}
\usetikzlibrary{arrows.meta, calc, positioning, shapes.geometric}
\usepackage{algorithm}
\usepackage{algpseudocode}
\usepackage{tabularx}
\usepackage{array}
\usepackage{float}
\newcolumntype{Y}{>{\raggedright\arraybackslash}X}
\usepackage[table]{xcolor}
\usepackage[most]{tcolorbox}
\tcbuselibrary{listings,skins}
% Neutral panel with teal accent for algorithms/notes
\newtcolorbox{algwithnotes}[1][]{enhanced,breakable,sharp corners,boxrule=0.6pt,
  colback=blue!3!white,colframe=black!20,borderline west={2pt}{0pt}{teal!70!black},
  title={#1},fonttitle=\bfseries}

% Listings style for nicer code blocks
\lstdefinestyle{codeblock}{%
  basicstyle=\ttfamily\small,
  breaklines=true,
  columns=fullflexible,
  keepspaces=true,
  showstringspaces=false,
  frame=single,
  framerule=0.5pt,
  backgroundcolor=\color{blue!5!white},
  rulecolor=\color{teal!70!black},
  numbers=left,
  numberstyle=\tiny,
  numbersep=16pt,
  xleftmargin=2.6em,
  framexleftmargin=2.0em
}
\lstset{style=codeblock}

% Theorem-like environments
\newtheorem{definition}{Definition}

\usepackage{placeins}
\usepackage{booktabs}
\usepackage{csquotes}
\usepackage[backend=biber, style=ieee]{biblatex}
\addbibresource{bib/references.bib}

% Only number and list up to sections (no subsection/subsubsection numbering in ToC or headings)
\setcounter{secnumdepth}{1}
\setcounter{tocdepth}{1}

\clearpairofpagestyles
\chead[Verifiable Data Transformations in IoT Environments – Ramón Felipe Kühne]{Verifiable Data Transformations in IoT Environments – Ramón Felipe Kühne}
% Unified footer for all page styles (plain and scrheadings)
\cfoot[TU Berlin, 2025 \quad | \quad \thepage]{TU Berlin, 2025 \quad | \quad \thepage}
\KOMAoptions{headsepline=.4pt}
\renewcommand*{\chapterpagestyle}{scrheadings}
\pagestyle{scrheadings}

\graphicspath{{./img/}{../data/visualizations/}}

%%%%%%%%%%%%%%%%%%%%%%%%%%%%%%%%%%%%%%%%%%%%%%%%%%%%%%%%%%%%%%%%%%%%%%%%%%%%%%%%%%%%%%%%%%%%%%%%%%%%
%%%%%%%%%%%%%%%%%%%%%%%%%%%%%%%%%%%%%%%%%%%%%%%%%%%%%%%%%%%%%%%%%%%%%%%%%%%%%%%%%%%%%%%%%%%%%%%%%%%%
%%%%%%%%%%%%%%%%%%%%%%%%%%%%%%%%%%%%%%%%%%%%%%%%%%%%%%%%%%%%%%%%%%%%%%%%%%%%%%%%%%%%%%%%%%%%%%%%%%%%

\begin{document}

\frontmatter
\thispagestyle{empty}
\begin{center}

\vspace*{1.4cm}
{\LARGE \textbf{Technische Universität Berlin}}

\vspace{0.5cm}

{\large Information Systems Engineering\\[1mm]}

Fakultät IV\\
Einsteinufer 17\\
10587 Berlin\\
https://www.tu.berlin/ise\\

\vspace*{1cm}

\includegraphics[width=4cm]{tu_logo.png}

\vspace*{1.0cm}

{\LARGE Thesis}\\

\vspace{1.0cm}
{\LARGE \textbf{Verifiable Data Transformations in IoT}}\\
\vspace*{0.3cm}
{\LARGE \textbf{Environments using Recursive zk-SNARKs}}\\
\vspace*{1.0cm}
{\LARGE Ramón Felipe Kühne}
\\
\vspace*{0.5cm}
Matriculation Number: 456119\\
\vspace*{1.0cm}

\textbf{First Examiner:} Prof. Dr.-Ing. Stefan Tai\\
\textbf{Second Examiner:} Prof. Dr. Sahin Albayrak

\vspace*{0.3cm}

\textbf{Supervised by}\\
Karl Wolf\\
Fabian Piper

\end{center}


    \thispagestyle{empty}
    \cleardoublepage
    
\input{misc/self-assertion}
    \thispagestyle{empty}
    \cleardoublepage

% Switch to arabic numbering for all following pages
\mainmatter
\pagenumbering{arabic}
\setcounter{page}{1}

\section*{Declaration of Authorship}
I hereby declare that I have written this thesis independently and have not used any sources or aids other than those stated. All passages taken directly or indirectly from the published or unpublished work of others have been identified as such. This thesis has not been submitted in substantially the same form to any other examination board and has not been published.

\clearpage

\section*{Abstract}

\clearpage

%%%%%%%%%%%%%%%%%%%%%%%%%%%%%%%%%%%%%%%%%%%%%%%%%%%%%%%%%%%%%%%%%%%%%%%%%%%%%%%%%%%%%%%%%%%%%%%%%%%%
%%%%%%%%%%%%%%%%%%%%%%%%%%%%%%%%%%%%%%%%%%%%%%%%%%%%%%%%%%%%%%%%%%%%%%%%%%%%%%%%%%%%%%%%%%%%%%%%%%%%
%%%%%%%%%%%%%%%%%%%%%%%%%%%%%%%%%%%%%%%%%%%%%%%%%%%%%%%%%%%%%%%%%%%%%%%%%%%%%%%%%%%%%%%%%%%%%%%%%%%%

\section*{Acknowledgements}

\clearpage

%%%%%%%%%%%%%%%%%%%%%%%%%%%%%%%%%%%%%%%%%%%%%%%%%%%%%%%%%%%%%%%%%%%%%%%%%%%%%%%%%%%%%%%%%%%%%%%%%%%%
%%%%%%%%%%%%%%%%%%%%%%%%%%%%%%%%%%%%%%%%%%%%%%%%%%%%%%%%%%%%%%%%%%%%%%%%%%%%%%%%%%%%%%%%%%%%%%%%%%%%
%%%%%%%%%%%%%%%%%%%%%%%%%%%%%%%%%%%%%%%%%%%%%%%%%%%%%%%%%%%%%%%%%%%%%%%%%%%%%%%%%%%%%%%%%%%%%%%%%%%%

\tableofcontents
\clearpage
\cleardoublepage
\listoffigures
\clearpage
\listoftables
\clearpage

%%%%%%%%%%%%%%%%%%%%%%%%%%%%%%%%%%%%%%%%%%%%%%%%%%%%%%%%%%%%%%%%%%%%%%%%%%%%%%%%%%%%%%%%%%%%%%%%%%%%
%%%%%%%%%%%%%%%%%%%%%%%%%%%%%%%%%%%%%%%%%%%%%%%%%%%%%%%%%%%%%%%%%%%%%%%%%%%%%%%%%%%%%%%%%%%%%%%%%%%%
%%%%%%%%%%%%%%%%%%%%%%%%%%%%%%%%%%%%%%%%%%%%%%%%%%%%%%%%%%%%%%%%%%%%%%%%%%%%%%%%%%%%%%%%%%%%%%%%%%%%

\chapter{Introduction}
\label{sec:intro}

\section{Motivation}
\label{sec:motivation}
IoT deployments such as smart home sensors, industrial monitors, and environmental sensing networks generate continuous high resolution time series data. To reduce communication and storage demands on resource constrained edge devices, this data is often aggregated in batches, for example via summation or averaging. Conventional aggregation lacks formal guarantees regarding data integrity and privacy \cite{SmartMeterPrivacySurvey2023, AStudyOnPrivacyPreserving2021}. Research has shown that even coarse patterns like hourly energy usage can expose private habits \cite{muellerGewinnungVerhaltensprofilenAm2010}. Aggregation pipelines that rely on unverified local computation are susceptible to tampering or omission, which undermines trust in reported values \cite{bohliPrivacyModelSmart2010aa}. This combination of emerging privacy threats and trust issues highlights the need for cryptographic verification mechanisms in IoT aggregation systems.

Global data production has exploded over the past decade. In 2010 approximately 2 zettabytes of data existed. By 2023 that number reached about 120 zettabytes. In 2024 it rose to approximately 147 zettabytes. Projections expect global data volume to grow further to 181 zettabytes by 2025 \cite{IDCExpect175, HowMuchData}. This dramatic increase magnifies the potential impact of large scale data breaches. In the healthcare sector alone the number of breaches reported to U.S. authorities reached 725 in 2023, exposing over 133 million records \cite{alderDecember2023Healthcare2024}.

The proliferation of IoT devices further accelerates data generation and increases privacy risk. Smart thermostats, smart meters, wearable health trackers, voice assistants, and environmental sensors continuously collect sensor data, often without users’ full awareness. These devices shape daily life and generate intimate behavioral insights.

Because massive volumes of data are generated every moment and breaches are escalating, ensuring the integrity and confidentiality of aggregated data has become critically important. This motivates the development of cryptographic methods that can verify aggregated IoT data without compromising user privacy.

\section{Problem Statement}
\label{sec:problemstatement}
The central problem of this thesis has two main aspects. First, existing aggregation methods do not provide formal integrity guarantees. In practice, users cannot confirm that published aggregates include all raw sensor readings nor detect whether any data were omitted or modified during processing. Second, although zkSNARKs allow confidential proofs of correctness, classical non recursive approaches become increasingly inefficient when used repeatedly for continuous streaming data. The core inefficiency stems from computational complexity, especially during witness generation, which can easily become a bottleneck on IoT hardware with limited resources \cite{ElHajj2024BenchmarkStudy}. In addition, many traditional zkSNARK protocols depend on a trusted setup and do not allow parallel processing, which limits their scalability in IoT use cases.

Recursive zkSNARKs present a promising alternative. They support proof chaining across batches, so that verification cost is amortized rather than repeated. Recent systems such as GENES demonstrate substantial improvements in proving time and verification latency through recursive proof composition. However, these improvements sometimes come with the trade off of larger overall proof sizes \cite{Genes2025EfficientRecursiveSnark}. Likewise, Zecale demonstrates how recursive aggregation can substantially reduce verification overhead while preserving privacy in blockchain contexts \cite{Rondelet2020Zecale}. Despite these theoretical advantages, the deployment of recursive zkSNARKs in constrained, privacy critical IoT environments has not yet been evaluated.

This research therefore targets a gap in current understanding by empirically establishing when recursive zkSNARKs offer a measurable advantage compared to classical zkSNARKs under realistic IoT conditions (hardware limits, privacy objectives, communication constraints). Our benchmarks compare recursive systems to non‑recursive implementations using identical data and report only measured latency, memory, and proof‑size results to produce actionable guidance for real‑world system designers.

\section{Research Questions}

Our research is guided by the following primary questions:

\begin{enumerate}
    \item Under which conditions is the use of recursive SNARKs beneficial?
    \item What added value do recursive SNARKs provide in the context of privacy?
    \item From which data volume or computational complexity onwards are recursive SNARKs more efficient?
    \item Can empirical measurements on realistic IoT setups determine when recursion becomes advantageous?
    \item What are the privacy-performance trade-offs in recursive vs. standard SNARK systems?
\end{enumerate}

\section{Contributions}

This thesis makes the following key contributions:

\begin{enumerate}
    \item \textbf{Nova Implementation}: Complete implementation of Nova recursive SNARKs optimized for IoT data processing
    \item \textbf{Empirical Validation}: Comprehensive resource-constrained IoT evaluation
    \item \textbf{Performance Analysis}: Detailed benchmarking across multiple scenarios
    \item \textbf{Practical Guidelines}: Decision frameworks for choosing appropriate proof systems
\end{enumerate}

\section{Thesis Structure}

\noindent The thesis is organized into nine chapters that map directly to the research questions and the evaluation pipeline:
\begin{enumerate}
    \item \textbf{Introduction} (\cref{sec:intro}): Motivation, problem statement, research questions, contributions, and chapter roadmap.
    \item \textbf{Background} (\cref{sec:background}): Task-relevant overview of IoT aggregation privacy risks and zero-knowledge systems; emphasis on concepts required to interpret the later crossover analysis.
    \item \textbf{Related Work} (\cref{chap:related-work}): Positioning within IoT privacy aggregation and zero-knowledge literature; highlights the gap this thesis addresses; core concepts and trade-offs of recursive vs. non-recursive zk-SNARKs relevant to this study.
    \item \textbf{Use Case} (\cref{chap:use-case}): Story-driven smart-home scenario (single-family house) motivating design choices.
    \item \textbf{System Architecture} (\cref{chap:system-architecture}): End-to-end architecture and components; actors; data flow and threat model; how the project is structured and executed.
    \item \textbf{Implementation} (\cref{chap:implementation}): Pipelines, tooling, measurement harness, reproducibility, and limitations.
    \item \textbf{Empirical Results and Analysis} (\cref{chap:empirical-results}): Integrated results covering crossover validation, temporal batching, sensitivity analysis, device-level performance, and practical selection guidelines.
    \item \textbf{Discussion} (\cref{chap:discussion}): Interpretation of findings, decision framework for practitioners, and threats to validity.
    \item \textbf{Conclusion \& Future Work} (\cref{chap:conclusion}): Summary of contributions and directions for future research.
\end{enumerate}

\noindent This structure keeps the narrative focused on the central objective: reporting strictly empirical advantages of recursive SNARKs in IoT settings. Each part either introduces a necessary concept, contributes a component of the methodology, or reports measured results that directly answer the research questions.

%%%%%%%%%%%%%%%%%%%%%%%%%%%%%%%%%%%%%%%%%%%%%%%%%%%%%%%%%%%%%%%%%%%%%%%%%%%%%%%%%%%%%%%%%%%%%%%%%%%%
%%%%%%%%%%%%%%%%%%%%%%%%%%%%%%%%%%%%%%%%%%%%%%%%%%%%%%%%%%%%%%%%%%%%%%%%%%%%%%%%%%%%%%%%%%%%%%%%%%%%
%%%%%%%%%%%%%%%%%%%%%%%%%%%%%%%%%%%%%%%%%%%%%%%%%%%%%%%%%%%%%%%%%%%%%%%%%%%%%%%%%%%%%%%%%%%%%%%%%%%%

\chapter{Background}
\label{sec:background}
\section{Privacy \& Data Aggregation in IoT}
Resource constrained devices in Internet of Things environments collect and transmit sensor data such as temperature, power usage or motion events. Aggregating this data can reduce communication load and storage overhead, but doing so without cryptographic guarantees can compromise data integrity or privacy. A review by Ali et al \cite{InayatAliPrivacyPreservingDataAggregation2018} shows that traditional data aggregation techniques may expose raw readings and remain vulnerable to inference or tampering, especially in constrained sensor networks. Solutions such as LiPI \cite{GoyalLiPI2022} propose lightweight data masking mechanisms, but they often trade off integrity verification or depend on trusted components. There is limited research on cryptographically verifiable aggregation tailored for resource limited IoT nodes, especially when continuous privacy preservation is required.

\section{Overview of Zero-Knowledge Proofs}
A zero-knowledge proof (ZKP) lets a prover convince a verifier that a statement is true without revealing any additional information beyond its truth \cite{goldwasserKnowledgeComplexityInteractive1985}. ZKPs exist in interactive and non-interactive forms.
zk-SNARKs are non-interactive arguments of knowledge with succinct proofs; in many constructions, proof size and verifier work are sublinear—often effectively constant—in the size of the computation, though they may depend on public input size and typically require a setup \cite{nitulescuZkSnarksAGentleIntroduction2021}. zk-SNARKs have seen prominent deployments, e.g., in Zerocash/Zcash \cite{bensassonZerocashDecentralizedAnonymous2014,hopwoodZcashProtocolSpecification}.
zk-STARKs are transparent (no trusted setup) and hash-based; they scale well and are plausibly post-quantum, but usually incur larger proofs and higher prover costs \cite{ben-sassonScalableZeroKnowledge2019}.
Bulletproofs provide short non-interactive proofs without trusted setup with logarithmic proof size for certain statements (e.g., range proofs); however, verification is generally more expensive than in SNARK systems for large circuits \cite{BulletproofsStanfordApplied}.
Other families (e.g., Sonic, Plonk/Halo variants) explore different trade-offs in universality, transparency, setup, and efficiency \cite{mallerSonicZeroKnowledgeSNARKs2019}.
A recent survey overviews applications and practical frameworks across domains \cite{ASurveyApplicationsZKP2024}.

\section{Verifiable Transformations in IoT Environments}

Verifiable transformations in IoT environments refer to cryptographic computations that process sensor data while maintaining both privacy and computational integrity guarantees. These transformations enable the verification of data processing correctness without revealing individual sensor readings, making them essential for privacy-preserving IoT aggregation systems.

\subsection{Definition and Requirements}

A verifiable transformation in IoT contexts must satisfy three fundamental requirements: (i) \emph{computational integrity}, ensuring that published results represent correct computations over authentic input data; (ii) \emph{privacy preservation}, preventing the disclosure of individual sensor readings beyond what is logically implied by the output; and (iii) \emph{verification efficiency}, enabling efficient proof verification even on resource-constrained devices.

\subsection{Circuit Logic and Cryptographic Components}

The circuit logic for IoT verifiable transformations typically involves several cryptographic components. First, \emph{device signature verification} ensures that each sensor reading originates from an authenticated IoT device using the device's public key. Second, \emph{computational integrity verification} proves that the aggregation function was applied correctly to the verified inputs. The circuit design distinguishes between \emph{public arguments} (device public keys, aggregated results, timestamps) and \emph{private arguments} (individual sensor readings, device private keys, intermediate computation states).

\subsection{General Transformation Types}

Our implementation demonstrates several classes of verifiable transformations applicable to IoT environments:

\begin{itemize}
\item \textbf{Range Validation}: Proving that sensor readings fall within acceptable bounds without revealing individual values
\item \textbf{Statistical Aggregation}: Computing sums, means, medians, or other statistics while maintaining input privacy
\item \textbf{Threshold Compliance}: Verifying that aggregated metrics meet predefined thresholds without exposing individual contributions
\item \textbf{Temporal Analysis}: Proving properties about data collected over time windows while preserving temporal privacy
\end{itemize}

These transformations form the foundation for privacy-preserving IoT data processing, enabling verifiable analytics while protecting individual device privacy.

\subsection*{Non-ZK baseline in this thesis}
Non-cryptographic processing can be faster, yet it reveals raw inputs during checking or depends on trust in the computing entity, which conflicts with our privacy requirement. In this thesis a non-zero-knowledge baseline is used only as a reference to contextualize raw execution time and output size. It does not serve as an alternative solution because it cannot provide both confidentiality and verifiable computational integrity at the same time \cite{goldwasserKnowledgeComplexityInteractive1985,katzIntroductionModernCryptography2007}.

\section{Recursive ZKPs and Aggregation}
Recursive zero knowledge proofs stack or fold multiple proofs into a single succinct result. This enables efficient and scalable verification especially in streaming or multi step computation settings where multiple sub proofs are generated.

\subsection*{Definition of aggregation in this thesis}
We use the term \emph{aggregation} narrowly to denote computations over sets or windows of readings that reduce raw data to concise statistics or validity results (e.g. range validation, sum/mean/median, min/max). In our scope, privacy-preserving aggregation must (i) reveal nothing about individual readings beyond what is logically implied by the output, and (ii) enable verifiers to check correctness without access to raw inputs. These requirements motivate zero-knowledge approaches \cite{goldwasserKnowledgeComplexityInteractive1985,katzIntroductionModernCryptography2007} and connect directly to our circuits and pipeline in \cref{chap:implementation}.

\subsection{Principles and Benefits}
The core idea of recursive ZKPs is to verify a proof inside another proof, thus composing multiple statements into an incrementally verifiable chain. This approach is formalized in theories such as incrementally verifiable computation and proof folding schemes. Nova introduced an efficient folding scheme that absorbs complexity into a relaxed R1CS representation, dramatically reducing per proof cost while maintaining succinct final proofs \cite{NovaFoldingIVC2023}. This makes recursion especially powerful when many steps must be verified sequentially.

\subsection{Frameworks: Halo, Nova, Plonky2}
Halo, introduced by Bowe et al in 2019, pioneered recursive SNARK designs that do not require a trusted setup. It supports cycles of elliptic curves and recursive proof composition transparently \cite{boweRecursiveProofComposition2019}. Nova builds on similar ideas through an efficient folding based proof aggregation strategy and achieves state of the art performance in proof generation and succinctness \cite{NovaFoldingIVC2023, PantheonNovaBenchmark}. Plonky2 is a zk-STARK based system optimized by Polygon Zero for recursive workloads. It uses custom gates and deep arithmetic constraints to enable recursion at scale with high proving speed \cite{Plonky2ZKM2025, AnalysisPlonky2Protocol, IntroducingPlonky2, MayaZKBlogAggregationSummary}. All three systems allow continual chaining of proofs and compression into a single final proof, reducing verification overhead in multi step or streaming use cases.

\section{Recursive vs. Non-Recursive zk-SNARKs in Resource-Constrained Environments}
\label{sec:zk-snark-comparison}

\subsection{Fundamentals: Difference Between Recursive and Non-Recursive zk-SNARKs}

zk-SNARKs (Zero-Knowledge Succinct Non-Interactive Arguments of Knowledge) allow one to prove the correctness of a computation using a short cryptographic proof without revealing the underlying data. A non-recursive zk-SNARK refers to a single proof for a specific computation or statement. In contrast, recursive zk-SNARKs allow multiple proofs or computation steps to be composed into each other. In recursion, the output of one zk-SNARK is used as part of the input for the next, resulting in a single final proof that attests to the correctness of all intermediate computations \cite{innovationBlockchainScalabilityGuide}. This is also known as incrementally verifiable computation (IVC): the prover produces a proof for each computation step that confirms both the correctness of that step and that the previous step was correctly verified \cite{bellachiaVerifBFLLeveragingZkSNARKs2025}. Through this composition, iterative or sequential computations can be securely chained.

A well-known example of recursive zk-SNARKs is Nova, which is based on a folding scheme. Nova folds a long computation into an ongoing recursive proof and only generates the final zk-SNARK at the end \cite{ElHajj2024BenchmarkStudy}. As a result, the expensive zk-SNARK generation occurs only once—regardless of how many steps were involved in the computation. Systems such as Halo or Nova have demonstrated that recursive zk-SNARKs can be built without a trusted setup, making them suitable for real-world applications \cite{ElHajj2024BenchmarkStudy}.

\subsection{Efficiency, Computation Cost, and Latency}

The primary efficiency difference lies in the trade-off between proof generation and verification. In non-recursive zk-SNARKs, generating a single proof is expensive, but verifying that proof is very fast (often milliseconds). However, if multiple zk-SNARKs must be verified (e.g., many individual proofs), the overall verification time scales linearly. Recursive zk-SNARKs aim to drastically reduce this verification overhead by aggregating all claims into a single proof \cite{innovationBlockchainScalabilityGuide}. Thus, the final verification time remains essentially constant, regardless of the number of individual steps or proofs involved.

On the proving side, recursive SNARKs introduce some overhead, since each new proof must verify the previous one, increasing the number of constraints. In traditional constructions (e.g., Groth16), verifying a SNARK inside a SNARK was costly. Modern systems like Nova optimize this by delaying the expensive zk-SNARK compression to the end \cite{ElHajj2024BenchmarkStudy}. Nova works in two stages: it first builds an ongoing recursive proof and then applies a final zk-SNARK compression. This final step incurs a fixed cost, regardless of how many steps were folded in. Hence, the final verification time remains constant, while the proof generation time increases roughly linearly with the number of steps. Latency may increase moderately, since the system waits until the end to compress the accumulated proofs.

Proof size is another major advantage. While a typical Groth16 proof is constant in size, producing many individual proofs results in linear growth in storage or transmission. Recursive SNARKs produce one final compact proof whose size is largely independent of the number of inputs \cite{innovationBlockchainScalabilityGuide}.

\subsection{Scalability}
Recursive zk-SNARKs are most beneficial when dealing with large-scale computations or proof aggregation. For small or one-time computations, a single non-recursive proof is often more efficient, as the recursive overhead may not be justified.

Empirical studies indicate that even at modest batch sizes (a few dozen proofs), recursion can become advantageous. For example, in a decentralized IoT setting, Nova required only \textasciitilde3.6 seconds to aggregate and verify 10 digital signatures, whereas a non-recursive method using Risc0 took \textasciitilde369 seconds—over 100$\times$ slower \cite{bojicburgosDecentralizedIoTData2024}. The gap grows with more inputs. Another study showed that Nova could verify 100 signatures in 7.1 seconds, whereas a previous method based on homomorphic encryption and ECDSA took over 50 seconds to verify just 64 signatures \cite{bojicburgosDecentralizedIoTData2024}. These results suggest that at batch sizes of a few dozen, recursive approaches can already be significantly more efficient.

Moreover, recursion reduces distributed verification overhead. Without recursion, each verifier must check all proofs. With recursion, only a single final proof needs to be verified. This makes the per-claim verification time negligible, since a constant cost is amortized over many claims \cite{bojicburgosDecentralizedIoTData2024}. The load is shifted from weak verifiers (e.g., IoT devices or smart contracts) to a single strong prover.

\subsection{Use in IoT and Smart-Home Scenarios}

IoT and smart-home environments impose strict constraints: sensors and embedded devices often have limited processing power, memory, and energy. zk-SNARK generation is typically too expensive to perform locally \cite{bojicburgosDecentralizedIoTData2024}. Even verification can overwhelm constrained devices. Therefore, many architectures follow a layered model with edge servers.

In this setup, IoT devices only collect and sign data. They then forward it to a nearby edge aggregator, which performs proof generation and aggregation \cite{bojicburgosDecentralizedIoTData2024}. Only the final proof or its hash is sent to a blockchain or central verifier. This eliminates the need for IoT devices to generate or verify SNARKs, saving energy and bandwidth.

Recursive zk-SNARKs are ideal for such scenarios, as they can aggregate continuous sensor streams into an ongoing proof. For instance, Nova has been used to aggregate and verify 100 sensor signatures into a single proof suitable for on-chain verification \cite{bojicburgosDecentralizedIoTData2024}. Verifying this proof took only \textasciitilde0.06\,s per signature (i.e., \textasciitilde6\,s total), even for low-powered verifiers.

Beyond signature verification, recursive SNARKs can prove compliance with rules over long periods, such as “no sensor exceeded a threshold for the past hour.” This streaming proof model allows incremental updates and compact final validation, ideal for constrained environments \cite{ElHajj2024BenchmarkStudy}.

Studies have even demonstrated recursive zk-SNARKs in advanced tasks like federated learning: each local training round and the global aggregation step are provably verified using Nova. In one setup, the global model proof took \textasciitilde81 seconds to generate and \textasciitilde0.6 seconds to verify \cite{bellachiaVerifBFLLeveragingZkSNARKs2025}. This shows that the cost is mostly on the proving side, which can be offloaded to strong devices.

\subsection{Summary and Implications for Architecture}
Recursive zk-SNARKs offer compelling benefits for scaling zero-knowledge applications in IoT scenarios. They enable aggregation of multiple computations or data streams into a single compact proof, which reduces memory, bandwidth, and verification cost—key concerns in resource‑constrained environments. Our system architecture (\cref{chap:system-architecture}) therefore places proving at an edge aggregator, defines batching policies that drive into empirically observed crossover regimes, and adopts metrics (proof time, verification time, proof size, and device load) that operationalize these trade‑offs. The following chapter translates these implications into a concrete architecture and methodology and specifies the evaluation setup used to validate them in practice.

%%%%%%%%%%%%%%%%%%%%%%%%%%%%%%%%%%%%%%%%%%%%%%%%%%%%%%%%%%%%%%%%%%%%%%%%%%%%%%%%%%%%%%%%%%%%%%%%%%%%
%%%%%%%%%%%%%%%%%%%%%%%%%%%%%%%%%%%%%%%%%%%%%%%%%%%%%%%%%%%%%%%%%%%%%%%%%%%%%%%%%%%%%%%%%%%%%%%%%%%%
%%%%%%%%%%%%%%%%%%%%%%%%%%%%%%%%%%%%%%%%%%%%%%%%%%%%%%%%%%%%%%%%%%%%%%%%%%%%%%%%%%%%%%%%%%%%%%%%%%%%

\chapter{Related Work}
\label{chap:related-work}

Research on privacy-preserving data verification in the Internet of Things (IoT) spans multiple domains, ranging from blockchain-based transparency mechanisms to lightweight edge-side proof systems. Zero-knowledge proofs, particularly zk-SNARKs, have recently emerged as a promising method to ensure computational integrity and confidentiality in distributed IoT environments. Prior studies have explored their integration with blockchain for decentralized trust, their adaptation to resource-constrained devices, and their use in recursive schemes to reduce verification overhead. This chapter reviews representative work across these categories and positions the present thesis within this landscape.

%\section{Blockchain-Based ZKP Solutions for IoT Data Integrity}
Several existing approaches combine zero-knowledge proofs with blockchain technology to secure IoT data and computations. Ramezan et al.\ introduce zk-IoT, a framework that uses zk-SNARKs on a blockchain platform to ensure integrity of IoT device operations \cite{ramezanZkIoTSecuringInternet2024}. Their system features zk-Devices which produce proofs (using functional commitments) for their firmware's execution, and it leverages a blockchain ledger to store and verify these proofs. This design establishes a trust layer among mutually distrustful devices and prevents tampering with sensor readings or code, even if devices are compromised. Notably, the zk-IoT experiments showed that proof generation on an IoT device can be completed in under one second, with verification in roughly 19\,ms, indicating practicality under IoT constraints. 

Other researchers have applied similar ideas in specific domains, for example, a privacy-preserving medical IoT system that uses a blockchain and zk-SNARKs to share healthcare sensor data securely \cite{scheibnerRevolutionizingMedicalData2020}. In such systems, the blockchain provides a transparent audit trail and decentralized trust, while zk-SNARK proofs guarantee that shared IoT data or aggregates of that data are correct and satisfy certain policies without exposing the raw readings. A common trade-off in these blockchain-based approaches is the additional overhead of on-chain verification and transaction costs. However, they achieve a strong trust model, since even untrusted third parties can verify IoT data integrity through the public blockchain. 

Recent work in federated learning follows a similar pattern. Awad et al.\ (VerifBFL, 2025) present a crowdsourced learning framework where end-to-end verifiability is guaranteed by logging proofs on a blockchain \cite{bellachiaVerifBFLLeveragingZkSNARKs2025}. Their system leverages Nova, a recursive SNARK, so that each round's model aggregation and accuracy check produces a proof that is compactly chained into a single proof for the entire training process. This use of recursion allows on-chain verification of many computation steps with minimal cost. Overall, blockchain-backed solutions demonstrate how zero-knowledge proofs can make IoT and edge data auditable and trustworthy in a decentralized manner, but they must balance this with performance considerations such as transaction latency and resource usage on small devices.

%\section{Off-Chain Verification in Resource-Constrained IoT Environments}
In contrast to blockchain-heavy designs, other works focus on local or edge-based verification to avoid extra infrastructure. zkStream by Swalens et al.\ is a recent framework that brings zero-knowledge proof guarantees to streaming data pipelines on the edge cloud without relying on a global ledger \cite{swalensZkStreamFrameworkTrustworthy2024}. Their scenario includes a smart home setting, such as smart energy meters, where raw sensor data is partially aggregated on an in-home edge device for efficiency. zkStream ensures two key properties: first, data confidentiality, meaning that sensitive sensor readings never leave the home in plain form; and second, computational integrity, ensuring that the utility company can trust that the edge device correctly executed the agreed aggregation logic. 

To achieve this, the framework combines digital signatures for data authenticity with zk-SNARK proofs attesting to correct computation over the signed sensor inputs. A crucial aspect is optimizing performance for the edge. The authors introduce techniques to reduce proving cost, such as offloading expensive cryptographic checks out of the circuit and using compact multi-signatures for batched sensor readings. As a result, zkStream can add trust guarantees with low latency overhead, and their prototype ran between 6.5× and 15× faster than a naive per-message proof approach. This indicates that even on constrained IoT hardware, carefully engineered zk-SNARK pipelines can be feasible. However, zkStream focuses exclusively on standard (non-recursive) SNARKs and does not explore when recursive proving becomes advantageous. Our work extends this line of research by empirically comparing standard and recursive pipelines under controlled resource constraints and identifying the crossover point where recursion overtakes standard proving in terms of time and proof size.

Earlier work on privacy in smart metering also explored off-chain zero-knowledge techniques. For instance, Danezis et al.\ (2011) showed that electricity usage totals could be proven correct via custom zero-knowledge protocols without revealing individual meter readings \cite{kursawePrivacyFriendlyAggregationSmartGrid2011}. Those systems did not use modern SNARKs, but they pioneered the idea of verifiable sensor aggregates for privacy. Today’s SNARK frameworks such as Circom, Groth16, and PLONK make it more practical to generate succinct proofs of sensor data transformations. The key difference of off-chain approaches is that a trusted verifier, such as the service provider or a regulator, checks the proofs directly instead of relying on global consensus. This avoids blockchain transaction costs and can be faster in a closed deployment, but it requires a clear trust model in which the verifier must be trusted not to leak or misuse any information it learns. In summary, non-blockchain IoT ZKP solutions prioritize efficiency and simplicity, proving data correctness within a localized trust boundary such as a smart home or single provider setting. They highlight that zero-knowledge proofs can enhance IoT privacy even without a distributed ledger, provided that the verifier role is well defined.

%\section{Recursive Proofs and Aggregation Efficiency}
A special focus of recent research, and of this thesis, is on recursive zk-SNARKs which allow multiple proof instances to be folded into one. This technique can drastically improve verification scalability for batch data or continuous streams. Several works have demonstrated the advantages of recursion in related contexts. Zecale (Rondelet et al., 2020) is an early proof aggregator that uses recursive composition of SNARKs to combine many proofs into a single succinct proof \cite{Rondelet2020Zecale}. A typical use case is on Ethereum, where Zecale can take numerous small proofs, such as sensor readings or transactions, and output one proof that the blockchain verifies, greatly reducing on-chain load. 

Similarly, Liu et al.\ (2025) introduce GENES, a new recursive SNARK protocol designed to improve performance without a trusted setup \cite{Genes2025EfficientRecursiveSnark}. GENES merges multiple R1CS constraint systems so that verifying many proofs becomes as cheap as verifying one, and it significantly cuts proving time compared to prior approaches, albeit with larger proof sizes. These advances illustrate the general trade-off in recursive proofs: they amortize verification cost and sometimes proving cost, but can produce larger proof artifacts. Until recently, recursive SNARK implementations were mainly discussed for blockchains or general verifiable computation. The IoT domain has only begun exploring them. 

One notable example is the VerifBFL system mentioned above, which applied Nova's recursive SNARK to federated learning on resource-limited edge devices. By using an incrementally verifiable computation approach, VerifBFL can prove a series of model updates and aggregations with a single succinct proof at the end of training. This idea is directly relevant to IoT sensor streams: rather than sending a proof for each sensor reading or each time window, an edge node could continuously update a rolling proof. The benefit is a fixed verification cost even as data accumulates, which is crucial for scalability. However, VerifBFL does not provide a direct comparison with non-recursive proving under identical circuit logic and resource constraints, nor does it quantify the crossover point where recursion becomes advantageous. The overhead of generating recursive proofs on constrained hardware and the conditions under which they outperform standard approaches remain open questions that our empirical study addresses. 

To our knowledge, no prior work has empirically pinpointed the crossover point where a recursive approach overtakes a non-recursive one under IoT conditions, and this gap is precisely what our study aims to address. In summary, recursive zk-SNARK technology such as Nova, PCD, and Halo promises to streamline verification of large or long-running IoT data computations. The current literature shows clear potential for recursion to improve efficiency, but it remains to be seen when those theoretical gains fully materialize in practical IoT deployments. This thesis builds on these insights to compare standard versus recursive proof pipelines in a realistic smart home setting.

%\section{Homomorphic Encryption and Blockchain-Based Aggregation}
A complementary line of research replaces zero-knowledge proofs with homomorphic encryption to achieve confidentiality in IoT or smart grid data aggregation. Wang et al.\ (2025) propose DA-MHEB, a privacy-preserving smart grid data aggregation scheme that combines multi-key homomorphic encryption with blockchain-based integrity guarantees \cite{wangPrivacyPreservingDataAggregation2025}. Their system introduces a three-layer architecture comprising smart meters, fog nodes, and cloud servers. Each smart meter encrypts its power consumption data under an individual key; fog nodes perform ciphertext aggregation and verification; and cloud servers collaboratively decrypt the aggregated results using a secret-sharing mechanism. To ensure tamper resistance, the aggregated ciphertexts are recorded on a blockchain, which provides transparency and immutability across regions. Experimental results show that DA-MHEB significantly reduces computational overhead compared to single-key homomorphic schemes while maintaining strong confidentiality and integrity guarantees.

Although both DA-MHEB and the present thesis address privacy-preserving data aggregation in distributed IoT environments, they differ fundamentally in their cryptographic foundation and trust model. DA-MHEB achieves privacy through encryption-based secrecy, data remain hidden because only aggregated ciphertexts are processed, whereas this thesis ensures privacy and integrity through verifiable computation: IoT devices or edge nodes generate zero-knowledge proofs that their reported aggregates are correctly computed from private inputs. In DA-MHEB, correctness of computation is implicit and relies on the honest behavior of the fog and cloud layers, while in our system, correctness is provably guaranteed via succinct proofs that can be verified independently by any party without re-execution.

Furthermore, the cryptographic trade-offs differ: multi-key homomorphic encryption supports additive aggregation directly but scales poorly for complex transformations or non-linear functions, whereas zk-SNARK circuits, as used in this thesis, can represent arbitrary computations while maintaining constant verification cost. Finally, DA-MHEB’s reliance on blockchain for auditability introduces on-chain latency and storage overhead, while our evaluation focuses on off-chain verification and on quantifying the performance crossover between standard and recursive zk-SNARKs (Nova) in an IoT setting. In essence, DA-MHEB exemplifies encryption-based privacy aggregation for energy data, whereas our approach targets verifiable, proof-based transformation with transparent performance metrics—complementary yet distinct directions toward trustworthy IoT data processing.

%%%%%%%%%%%%%%%%%%%%%%%%%%%%%%%%%%%%%%%%%%%%%%%%%%%%%%%%%%%%%%%%%%%%%%%%%%%%%%%%%%%%%%%%%%%%%%%%%%%%
%%%%%%%%%%%%%%%%%%%%%%%%%%%%%%%%%%%%%%%%%%%%%%%%%%%%%%%%%%%%%%%%%%%%%%%%%%%%%%%%%%%%%%%%%%%%%%%%%%%%
%%%%%%%%%%%%%%%%%%%%%%%%%%%%%%%%%%%%%%%%%%%%%%%%%%%%%%%%%%%%%%%%%%%%%%%%%%%%%%%%%%%%%%%%%%%%%%%%%%%%

\chapter{Use Case}
\label{chap:use-case}

This work motivates verifiable, privacy-preserving transformation by means of a familiar, everyday setting: a single-family smart home in which multiple IoT devices continuously collect measurements. Typical endpoints include a smart heating unit, lighting sensors and switches, and metered appliances such as refrigerator, washing machine, and oven. All of these consume electricity and emit telemetry. Prior research has shown that detailed power-use patterns can reveal which appliances are running and whether a home is occupied or not \cite{muellerGewinnungVerhaltensprofilenAm2010}. Appliance usage and presence patterns can often be inferred from aggregate meter data alone, thereby revealing routines and activities of residents \cite{muellerGewinnungVerhaltensprofilenAm2010,mckenna2012privacy,SmartMeterPrivacySurvey2023}. Such inferences illustrate that naively exporting raw time series to external parties poses a material privacy risk.

 \begin{figure}[H]
  \centering
  % --- Spülmaschine ----------------------------------------------------------
  \begin{subfigure}[t]{0.49\textwidth}
    \centering
    \begin{tikzpicture}
      \begin{axis}[
        width=\linewidth, height=5.0cm,
        xmin=0, xmax=120, ymin=0, ymax=2500,
        xlabel={Time}, ylabel={Power [W]},
        xtick={0,30,60,90,120},
        xticklabels={13{:}00,13{:}30,14{:}00,14{:}30,15{:}00},
        ymajorgrids, xmajorgrids, line width=0.9pt,
        tick align=outside
      ]
        % Schematisches Lastprofil Spülmaschine: zwei Heizphasen + Niedriglast
        \addplot+[no marks] coordinates {
          (0,20) (2,2200) (13,2000) (15,2150) (19,2100) (22,2125) (25,2050) (28,2000) 
          (30,60) (40,90)(50,60)(60,90)(70,60)(80,90) (90,60) (95,30) (98,40) 
          (100,2100) (111,2050) (115,2100) (118,2000) (120,40)
        };
      \end{axis}
    \end{tikzpicture}
    \caption{Dishwasher}
    \label{fig:dishwasher-profile}
  \end{subfigure}
  \hfill
  % --- Kühlschrank -----------------------------------------------------------
  \begin{subfigure}[t]{0.49\textwidth}
    \centering
    \begin{tikzpicture}
      \begin{axis}[
        width=\linewidth, height=5.0cm,
        xmin=0, xmax=180, ymin=0, ymax=50,
        xlabel={Time}, ylabel={Power [W]},
        xtick={0,30,60,90,120,150,180},
        xticklabels={1{:}30,2{:}00,2{:}30,3{:}00,3{:}30,4{:}00,4{:}30},
        ymajorgrids, xmajorgrids, line width=0.9pt,
        tick align=outside
      ]
        % Schematisches Lastprofil Kühlschrank: periodische Kompressorzyklen
        \addplot+[no marks] coordinates {
          (0,12) (8,12) (10,40) (20,40) (22,12)
          (25,18)
          (30,12) (32,40) (42,40) (44,12)
          (52,12) (54,40) (64,40) (66,12)
          (69,25)
          (74,12) (76,40) (86,40) (88,12)
          (96,12) (98,40) (108,40) (110,12)
          (118,12) (120,40) (130,40) (132,12)
          (140,12) (142,40) (152,40) (154,12)
          (162,12) (164,40) (174,40) (176,12) (180,12)
        };
      \end{axis}
    \end{tikzpicture}
    \caption{Refrigerator}
    \label{fig:fridge-profile}
  \end{subfigure}

  \vspace{0.25em}
  \caption{load profiles illustrating one device each \cite{muellerGewinnungVerhaltensprofilenAm2010}}
  \label{fig:schematic-load-profiles}
 \end{figure}
 \FloatBarrier

\noindent
\Cref{fig:dishwasher-profile} shows a typical dishwasher load profile. Two extended plateaus at roughly \SI{2}{kW} mark phases in which water is taken in and heated and the circulation pump runs at high power (first around 13{:}00–13{:}30, then again near 14{:}30–15{:}00). Between these plateaus the appliance remains active but at a much lower draw, consistent with recirculating and filtering the water that is already in the dishwasher. The brief returns to the near-baseline level indicate short idle intervals between internal stages of the program.

\noindent
In contrast, \Cref{fig:fridge-profile} shows the behavior of a refrigerator. The refrigerator cools at about \SI{40}{W} and remains off near the baseline in between, driven by a thermostat that maintains the target temperature. Two short anomalies are visible (around 1{:}45 and 2{:}40): the power briefly increases. The refrigerator was opened, which caused the light inside to come on and the refrigerator to cool a little because heat had entered.

\noindent
These contrasting signatures illustrate how appliance-specific behavior can be inferred from raw time-series power data.

  \begin{figure}[H]
     \centering
     \begin{tikzpicture}
       \begin{axis}[
         width=\linewidth, height=5.0cm,
         xmin=360, xmax=1440, ymin=0, ymax=5000,
         xlabel={Time}, ylabel={Power [W]},
         xtick={360,480,600,720,840,960,1080,1200,1320,1440},
         xticklabels={06{:}00,08{:}00,10{:}00,12{:}00,14{:}00,16{:}00,18{:}00,20{:}00,22{:}00,24{:}00},
         ymajorgrids, xmajorgrids, line width=0.9pt,
         tick align=outside
       ]
         % Detailed combined daily load (single summed trace), 00:00--24:00
         % Units: minutes since 00:00 on x-axis, Watts on y-axis
         \addplot+[no marks, color=blue!70!black, very thick] coordinates {
           % Night baseline (ensure left-edge point at 06:00 is present)
           (0,40) (360,40) (405,40)
           % Morning routine: lights on
           (405,190) (425,190)
         % Shower begins (instant heater)
           (425,3790) (430,3790)
           % Shower continues (no kettle)
           (430,3790) (432,3790)
           % Toaster overlaps for ~1 min (shower + toaster)
           (432,4590) (433,4590)
           % Toaster ends, shower continues
           (433,3790) (435,3790)
           % Shower ends soon after
           (435,3790) (440,3790)
           % Only lights remain
           (440,190) (445,190)
           % Coffee machine
           (445,1190) (450,1190)
           % Hairdryer
           (450,1390) (455,1390)
           % Lights wrap-up
           (455,190) (465,190)
           % Leave for work: baseline
           (465,40) (600,40)    
           % Midday baseline (no kettle)
           (750,40) (1020,40)
           % Arrive home
           (1020,40) (1050,40)
           % Evening lights on
           (1050,240) (1080,240)
           % Cooking: oven + cooktop
           (1080,3740) (1100,3740)
           % Oven continues
           (1100,2740) (1120,2740)
           % Short microwave burst
           (1120,240) (1125,240)
           (1125,1240) (1130,1240)
           (1130,240) (1140,240)
           % Early evening
           (1140,240) (1170,240)
           % TV + lights
           (1170,360) (1260,360)
           % Dishwasher cycle high phase
           (1260,2460) (1280,2460)
           % Dishwasher low phase
           (1280,440) (1320,440)
           % Dishwasher second high phase
           (1320,2460) (1340,2460)
           % Cooldown
           (1340,400) (1350,400)
           % Lights/TV off, night baseline
           (1350,40) (1440,40)
         };
       \end{axis}
     \end{tikzpicture}
     \caption{combined detailed household load for one day}
     \label{fig:daily-combined}
   \end{figure}
   
\noindent
\Cref{fig:daily-combined} shows a typical household’s combined load over one day. The profile is quiet overnight at a low baseline, then rises sharply between about 06{:}00 and 08{:}00—consistent with wake-up routines such as lights, coffee machine, shower and short kitchen use. Afterward the load collapses back to baseline for most of the workday, indicating the home is likely unoccupied. In the early evening the power ramps up again as occupants return (lighting, cooking, entertainment). Notably, a distinctive high, rectangular plateau appears after 20{:}00; this matches the dishwasher signature characterized in \Cref{fig:dishwasher-profile}. The key observation is that, even in an aggregate household trace, individual appliance fingerprints remain visible. Over longer periods, such patterns reveal daily rhythms (when people are home or away) and specific activities, underscoring the privacy risk of sharing raw time-series data and motivating our zero-knowledge approach that proves aggregate properties without disclosing full traces.

We consider the conventional pipeline in which devices forward measurements to a household smart hub, which then relays data to the energy provider for billing. For the sake of worst-case analysis, we assume continuous telemetry export to the provider. Under this assumption, the provider gains access to fine-grained data capable of supporting detailed behavioral inference as shown in \Cref{fig:daily-combined} \cite{mckenna2012privacy,SmartMeterPrivacySurvey2023}. 

Our design replaces the sharing of raw readings with zero-knowledge proofs over authenticated device data. In our setting, the smart hub proves that the household’s energy usage lies within an acceptable range defined by the provider’s policy. The utility verifies the proof and learns only the released aggregates or a compliance flag. 

Instead of streaming fine-grained traces, the household sends only power totals every two to four weeks. In between, the energy provider still receives regular zero-knowledge proofs showing that the readings stayed within the agreed bounds and that the aggregates were computed correctly. No raw data leaves the home, detailed behavior remains private, and the provider can still bill accurately from the verified aggregates.

%%%%%%%%%%%%%%%%%%%%%%%%%%%%%%%%%%%%%%%%%%%%%%%%%%%%%%%%%%%%%%%%%%%%%%%%%%%%%%%%%%%%%%%%%%%%%%%%%%%%
%%%%%%%%%%%%%%%%%%%%%%%%%%%%%%%%%%%%%%%%%%%%%%%%%%%%%%%%%%%%%%%%%%%%%%%%%%%%%%%%%%%%%%%%%%%%%%%%%%%%
%%%%%%%%%%%%%%%%%%%%%%%%%%%%%%%%%%%%%%%%%%%%%%%%%%%%%%%%%%%%%%%%%%%%%%%%%%%%%%%%%%%%%%%%%%%%%%%%%%%%

\chapter{System Architecture}
\label{chap:system-architecture}

This chapter presents the system architecture for privacy-preserving verification of IoT computations. The goal is to enable verification of range checks and aggregate properties of sensor data without revealing the raw measurements. We use zk-SNARKs for standard proving and recursive zk-SNARKs for batch wise verification of many steps as a single succinct proof. The chapter first introduces the actors and their trust relationships, then specifies keys and credentials, and finally describes the minimal interfaces (witness and policy) and the two proving modes. Implementation details such as concrete commands, scripts and build steps are deferred to the Implementation chapter.


\begin{figure}[H]  % oder [!htb] wenn du es nicht strikt erzwingen willst
\centering
\resizebox{0.98\textwidth}{!}{%
\begin{tikzpicture}[
    node distance=2.2cm and 3.0cm,
    actor/.style={draw, rounded corners=8pt, align=center, minimum width=3.6cm, minimum height=0.95cm, fill=gray!8, font=\scriptsize},
    center/.style={actor, fill=blue!20, text=white, font=\bfseries\scriptsize},
    arr/.style={-{Stealth[length=2.2mm,width=1.2mm]}, thick}
  ]
  % Center
  \node[center] (edge) {Edge Orchestrator};
  
  % Ring
  \node[actor, above=of edge]            (pki)  {Key Registry};
  \node[actor, left=of edge]             (dev)  {IoT Device};
  \node[actor, right=of edge]            (prov) {Prover};
  \node[actor, below left=1.6cm and 1.8cm of edge] (ver)  {Verifier};
  \node[actor, below right=1.6cm and 1.8cm of edge] (cons) {Consumer};
  
  % Flows
  \draw[arr] (dev) -- node[above, font=\scriptsize]{sensor readings + signature} (edge);
  \draw[arr] (pki) -- node[right, font=\scriptsize]{device credential} (edge);
  \draw[arr] (edge) -- node[above, font=\scriptsize]{witness + policy} (prov);
  \draw[arr] (prov) -- node[above, font=\scriptsize]{standard/recursive proof} (ver);
  \draw[arr] (ver) -- node[above, font=\scriptsize]{verified result} (cons);
\end{tikzpicture}
}
  \caption{System architecture and interactions}
  \label{fig:usecase-architecture}
\end{figure}



\section{Actors and Trust Model}
The architecture includes six roles that interact as shown in Figure~\ref{fig:usecase-architecture}. The IoT Device generates sensor readings and signs each payload with a device key that never leaves the device. The Key Registry, acting as issuer and certification authority, issues a device credential that binds the device identity to its public key. The Edge Orchestrator receives sensor readings and signatures, verifies the certificate and each signature, and prepares the public policy and the private witness for proving. The Prover produces a proof for the prepared statement, either as a standard zk-SNARK proof or as a recursive zk-SNARK proof. The Verifier checks the received proof with the verification key, and the Consumer receives only a verified result. We assume the certification authority is trusted by the Edge Orchestrator and the Verifier, the device private key remains on the device, the device credential is issued by the certification authority, the proving key stays with the Orchestrator, and the verification key is distributed to verifiers.

\noindent\textit{Trust domains and boundaries.} Figure~\ref{fig:usecase-architecture} overlays domains to make trust explicit and indicates confidentiality: the \emph{device domain} is high confidentiality and is trusted for private-key custody; the \emph{PKI/Key Registry} is a trusted root for authenticating devices; the \emph{edge domain} (orchestrator) is honest-but-curious—it authenticates inputs and prepares witness/policy but must not learn raw readings beyond what is necessary; the \emph{prover domain} is untrusted compute whose outputs are accepted only via zk proofs; the \emph{verifier domain} trusts only the verification key and PKI root; and the \emph{consumer domain} relies solely on the verifier’s checked result. Highlighted markers indicate where authentication (certificates, signatures) and proof verification enforce trust boundaries.

\section*{Use Case Revisited}
Anchored in the smart-home story (\cref{chap:use-case}), this architecture ensures that the heating unit’s fine-grained readings stay private while the utility receives succinct, verifiable aggregates. Recursive proving supports a single proof per billing period; standard proving supports low-latency per-window checks. The choice is driven by the operating constraints and goals laid out in the use case.

\section{Key Management}
Device identity is established by generating the keypair on the device and obtaining a credential issued by the certification authority. In our evaluation a local certification authority is created to keep runs reproducible and the resulting artifacts are managed by the orchestration environment. For the proof system the proving key is created locally by the orchestration role during setup and it is not shared. The verification key is exported and provided to the verification role for proof checking.

\section{Data and Control Flow}
Data and control follow the simple pattern shown in Figure~\ref{fig:usecase-architecture}. The IoT Device emits sensor readings together with a signature. The certification authority provides an authority root and a device credential to enable authentication. The Edge Orchestrator verifies both, prepares public policy and private witness, and triggers proving. The Prover returns either a standard proof or, if recursion is used, a compressed proof. The Verifier checks the proof with the verification key and the Consumer receives only a verified result.

\section{Verifiable Transformations}
We define a general transformation \(f\) that processes private readings under a public policy. Public inputs encode the policy and minimal metadata. Private inputs hold only the sensor readings or aggregates. The circuit guarantees the computational integrity of \(f\) on the private inputs under the supplied policy, so a verifier can rely on the result without seeing the readings. Typical instances are range or threshold checks, minimum/maximum over a set and the median over a window.

\section{Proving Pipelines}
The system supports two proving modes. The standard mode emits a zk‑SNARK per reading and verification effort grows with the number of readings. The recursive mode folds many steps and can produce a single compressed proof for a batch, which keeps verification essentially constant while shifting work to proving. Detailed tooling and commands are described in the Implementation chapter~\ref{chap:implementation}.

\section{Batch size}
Batching defines how many readings a single recursive step consumes. A small step size reduces work per step and can reduce memory pressure at the edge. It also increases the number of steps for a given dataset. The total proving cost is a combination of per step work and the final compression. A large step size reduces the number of steps and can improve end to end time once the prover amortizes the fixed costs of recursion and compression. Verification remains constant in both cases since the consumer checks a single proof for the entire batch. Padding to a multiple of the step size is part of the design and does not change the public claim. Very large step sizes can raise peak memory per step and increase latency until the final proof is available. If the input rate is irregular, larger step sizes can increase padding, which wastes capacity and shifts the effective input mix. A failure or restart also invalidates more work when steps are large. In practice a moderate step size is often ideal. It keeps per step memory bounded while reducing the total number of steps enough to move the time crossover toward smaller input sizes. The Results and Analysis chapter~\ref{chap:empirical-results} reports this effect explicitly.

\section{Security Properties and Assumptions}
The system keeps raw readings private and shares only proofs together with public policy parameters. The proofs ensure the computational integrity of the transformation. Authenticity follows from signature checks against the device certificate issued by a trusted authority. The proving key is held by the orchestrator, while the device’s private key remains on the device. We assume that verifiers are pre-configured with the issuer’s public key, a correct ZoKrates setup, and an honest-but-curious verifier \cite{malekzadehHonestbutCuriousNetsSensitive2021}.

\section{Performance Considerations}
We report proving time, verification time, proof size and the memory footprint of the process. The standard pipeline offers fine grained proofs but requires many verifications as the number of readings grows. The Nova pipeline amortizes verification to a nearly constant cost while adding overhead on the proving side.

\section{Limitations and Extensions}
Networking security such as TLS, key rotation, revocation via CRL or OCSP and Merkle based data binding are outside the scope of this chapter. A non zero knowledge baseline that computes the same transformations locally and ships only results can provide a reference for raw execution time and artefact size, but it does not meet the privacy requirement since raw inputs must be trusted or revealed during checking. Future work can bind entire reading sets through commitment roots, integrate certificate revocation and extend the circuit family with a sum in range billing scheme.


\begin{figure}[H]
\centering
\resizebox{0.98\textwidth}{!}{%
\begin{tikzpicture}[
  scale=0.8, transform shape,
  lane/.style={draw, fill=gray!8, minimum width=3.7cm, minimum height=26cm},
  task/.style={draw, fill=white, rounded corners, minimum width=3.1cm, minimum height=0.95cm, font=\scriptsize, align=center},
  opt/.style={task, dashed},
  note/.style={draw, fill=gray!5, rounded corners, font=\scriptsize, align=left, minimum width=5.0cm},
  arrow/.style={-{Stealth[length=2.4mm,width=1.4mm]}, thick}
]
\node[lane] (L0) at (-4,0) {}; \node[above] at (L0.north) {Key Registry};
\node[lane] (L1) at ( 0,0) {}; \node[above] at (L1.north) {IoT Device};
\node[lane] (L2) at ( 4,0) {}; \node[above] at (L2.north) {Edge Orchestrator};
\node[lane] (L3) at ( 8,0) {}; \node[above] at (L3.north) {Prover};
\node[lane] (L4) at (12,0) {}; \node[above] at (L4.north) {Verifier};
\node[lane] (L5) at (16,0) {}; \node[above] at (L5.north) {Consumer};

\node[task] (d0) at (0,12.0)  {Generate device keypair\\(private stays on device)};
\node[task] (k1) at (-4,10.5) {Issue device credential (certification authority)};
\node[task] (k2) at (-4,9.0) {Publish authority root and device credential};

\node[task] (d1) at (0,7.5)  {Generate sensor data};
\node[task] (d2) at (0,6.0)  {Sign reading};

\node[task] (a0) at (4,4.5)  {Load authority root and device credential};
\node[task] (a1) at (4,3.0)  {Verify device certificate};
\node[task] (a2) at (4,1.5)  {Verify device signature (reading)};
\node[task] (a3) at (4,0.0)  {Manage proving key \\ Export verification key};
\node[task] (a4) at (4,-1.5)  {Apply computation + ZK proof};
\node[task]  (a5) at (4,-3.0) {Form steps (batching)};

\node[task] (p1) at (8,-4.5) {Generate standard proof};
\node[task]  (p2) at (8,-6.0) {Generate recursive proof};

\node[task] (v0) at (12,-7.5) {Import verification key};
\node[task] (v1) at (12,-9.0) {Verify standard proof};
\node[task]  (v2) at (12,-10.5) {Verify recursive proof};

\node[task] (c1) at (16,-12.0) {Receive verified result};

\draw[arrow] (d0) to[out=180, in=90, looseness=1.2] node[above,font=\scriptsize]{1} (k1);
\draw[arrow] (k1) -- node[left,font=\scriptsize] {2} (k2);
\draw[arrow] (k2) to[out=-90, in=180, looseness=1.5] node[below, font=\scriptsize]{3} (a0);

\draw[arrow] (d1) -- node[right,font=\scriptsize] {4} (d2);
\draw[arrow] (d2) to[out=-90, in=180, looseness=1.5] node[left, font=\scriptsize]{5} (a2);

\draw[arrow] (a0) -- node[right,font=\scriptsize] {6} (a1);
\draw[arrow] (a1) -- node[right,font=\scriptsize] {7} (a2);
\draw[arrow] (a2) -- node[right,font=\scriptsize] {8} (a3);
\draw[arrow] (a3) -- node[right,font=\scriptsize] {9} (a4);

\draw[arrow] (a4) to[out=0, in=90, looseness=1.5] node[right, font=\scriptsize]{10} (p1);
\draw[arrow] (p1) to[out=0, in=90, looseness=1.5] node[right, font=\scriptsize]{11} (v0);
\draw[arrow] (v0) -- node[right,font=\scriptsize] {12} (v1);
\draw[arrow] (v1) to[out=0, in=90, looseness=1.5] node[right, font=\scriptsize]{13} (c1);

\draw[arrow] (a4) -- node[right,font=\scriptsize] {14} (a5);
\draw[arrow] (a5) to[out=-90, in=180, looseness=1.5] node[left, font=\scriptsize]{15} (p2);
\draw[arrow] (p2) to[out=-90, in=180, looseness=1.5] node[left, font=\scriptsize]{16} (v2);
\draw[arrow] (v2) to[out=-90, in=180, looseness=1.5] node[below, font=\scriptsize]{17} (c1);
\end{tikzpicture}
}
\caption{System Architecture with Key Registry, proving, and verification flows.}
\label{fig:system-overview}
\end{figure}

The figure illustrates the complete workflow from device onboarding to proof generation and verification, following the numbered arrows. The IoT device first generates its keypair (1), keeping the private key securely on the device. The key registry, acting as a certification authority, then issues a device credential (2). The authority’s root information and the credential are published so that the edge can import them (3).

Once the device starts taking measurements, it produces a sensor reading (4), signs the reading (5), and sends the signed payload to the edge orchestrator. At the edge, the authority root and the device credential are loaded (6). The system then validates the device certificate (7) and verifies the device’s signature on the reading (8). After successful verification, the proving material is prepared (9). This includes keeping the proving key locally, exporting the verification key for external verifiers, and assembling the public policy together with the private witness that contains the readings or aggregates required by the circuit.

From this point, the execution can proceed in two equivalent modes that prove the same statement over the same inputs. In the standard mode, a standard zk-SNARK proof is generated (10). The verifier obtains the verification key (11), verifies the proof (12), and the consumer receives the verified result (13). In the recursive mode, the process groups readings into batches according to the configured batch size, padding the final batch if necessary without changing the public claim (14). A recursive zk-SNARK proof is then generated by folding the steps and compressing the accumulated state into a single succinct proof (15). The proof is verified (16), and the verified result is provided to the consumer (17).

Throughout the process, device certificates and signatures ensure authenticity, while zero-knowledge proofs guarantee computational integrity and privacy. The verifier learns only the public policy and the final claim, never the raw sensor readings.

%%%%%%%%%%%%%%%%%%%%%%%%%%%%%%%%%%%%%%%%%%%%%%%%%%%%%%%%%%%%%%%%%%%%%%%%%%%%%%%%%%%%%%%%%%%%%%%%%%%%%%%%%%%%%%%%%%%%%%%%%%%%%%%%%%%%%%%%%%%%%%%%%%%%%%%%%%%%%%%%%%%%%%%%%%%%%%%%%%%%%%%%%%%%%%%%%%%%%%%%%%%%%%%%%%%%%%%%%%%%%%%%%%%%%%%%%%%%%%%%%%%%%%%%%%%%%%%%%%%%%%%%%%%%%%%%%%%%%%%%%%%%%%%%%%%%%%%%%%%%%%%%%%%%%%%%%%%%%%%%%%%%%%%%%%%%%%%%%%%%%%%%%%%%%%%%%%

\chapter{Implementation}
\label{chap:implementation}
This chapter maps the architecture to a concrete, reproducible system. It explains the evaluation environment and limits, the roles and their responsibilities during a run, the proof modes and the data products that feed the analysis in the next chapter.

\section{Environment and Hardware Profile}
\label{sec:env-hw}

We executed all experiments under a resource-constrained profile intended to simulate realistic IoT edge environments. 
Specifically, we used Docker containers with enforced limits on CPU and memory to approximate the hardware characteristics of IoT-devices. For example:

\begin{itemize}
  \item \textbf{CPU}: fixed 0.5 logical core
  \item \textbf{RAM}: fixed 1\,GB
\end{itemize}

These settings are chosen based on typical Raspberry Pi-class device specifications, which often provide 0.5-4 GB RAM and single-to-quad-core CPUs in edge deployments. For example, Gupta \& Nahrstedt \cite{guptaPerformanceCharacterizationContainers2025} empirically evaluate similar IoT devices and show that Docker containers on hardware with ~1 GB RAM suffer measurable overheads in latency and I/O under constrained CPU/RAM settings.\\

\noindent Host platform (for emulation):
\begin{table}[H]
  \centering
  \begin{tabular}{ll}
    \hline
    Component & Specification \\
    \hline
    CPU & AMD Ryzen 7 7800X3D (8 cores, 16 threads; base \SI{4.2}{GHz}) \\
    RAM & \SI{32}{GB} DDR5 @ \SI{6000}{MT/s} (2\,\texttimes{} DIMM) \\
    GPU & NVIDIA GeForce RTX 4070 SUPER (12\,GB VRAM) \\
    Virtualization & Windows 11 + WSL2 (Ubuntu kernel 6.6.87.2) \\
    \hline
  \end{tabular}
  \caption{Host platform used to simulate IoT-like environments under resource constraints via Docker/cgroups.}
\end{table}

The implementation follows a compact end to end flow. All inputs and outputs use JSON, and each payload is converted to a single, well defined JSON form before signing so that signatures are deterministically verifiable. A separate serializer is not required because the verifier reconstructs the exact same byte sequence. For each IoT device an Ed25519 key pair is generated and a local certification authority issues a device credential bound to the device key. Before every run the orchestrator validates issuer and subject linkage, the CA signature, and the certificate validity window. Only then are signatures over JSON accepted.

Public \emph{policy} parameters, for example lower and upper bounds, are treated as public inputs, while the readings form the \emph{private witness}. In the standard proving path valued readings are scaled to integers (e.g., \(\times 100\)) and passed as \([\text{MIN}, \text{MAX}, \text{READING}]\), where MIN and MAX define the acceptable range and READING is the private sensor value. In the recursive path, values are grouped into fixed size batches and serialized alongside an initial accumulator. The prover folds these batches and compresses the final accumulator to a succinct proof.

The measurement pipeline is intentionally minimal: load the data, validate credentials, build policy and witness, produce a proof, verify it, and persist artifacts and metrics. Concretely, the standard path loads an input \(x\), scales it to an integer \(x'\), computes a witness for \([\text{MIN}, \text{MAX}, x']\), generates the proof, verifies it, and records witness, prove, and verify timings and sizes. The recursive path splits values into batches of step size \(b\) (padding with zeros if needed), writes the initial state and step list, runs proving and optional compression followed by verification, and records total, prove, and verify time and the final proof size.

The orchestration logic and measurement harness are implemented in Python 3.10. The circuits for both standard and recursive proving are written in ZoKrates 0.8.8, a domain specific language for zk-SNARKs that compiles to rank-1 constraint systems. The standard pipeline uses the BN128 curve with Groth16, while the recursive pipeline uses the Pallas curve with Nova folding. BN128 is chosen for the standard path because it is the default and most widely deployed curve for Groth16 in production systems, offering mature tooling and well understood security properties. Nova requires a cycle of elliptic curves to enable efficient recursive proof verification, and Pallas is specifically designed for this purpose. The Pallas curve supports the folding scheme that allows Nova to verify one proof inside another without prohibitive overhead. While it would be technically possible to use Pallas for both pipelines, doing so would deviate from standard Groth16 practice and potentially introduce artifacts unrelated to recursion. Despite the different curves, both pipelines implement identical circuit logic for range validation, ensuring that the comparison measures only the structural differences between standard and recursive proving.

%%%%%%%%%%%%%%%%%%%%%%%%%%%%%%%%%%%%%%%%%%%%%%%%%%%%%%%%%%%%%%%%%%%%%%%%%%%%%%%%%%%%%%%%%%%%%%%%%%%%
%%%%%%%%%%%%%%%%%%%%%%%%%%%%%%%%%%%%%%%%%%%%%%%%%%%%%%%%%%%%%%%%%%%%%%%%%%%%%%%%%%%%%%%%%%%%%%%%%%%%
%%%%%%%%%%%%%%%%%%%%%%%%%%%%%%%%%%%%%%%%%%%%%%%%%%%%%%%%%%%%%%%%%%%%%%%%%%%%%%%%%%%%%%%%%%%%%%%%%%%%

% Requirements (load once in the preamble):
% \usepackage{graphicx}
% \usepackage{booktabs}
% \usepackage{siunitx}
% \usepackage{hyperref}
% \usepackage{caption}
% \sisetup{round-mode=places,round-precision=2,detect-weight=true,detect-inline-weight=math}
% =========================
% RESULTS CHAPTER (custom captions per run)
% =========================

\chapter{Results and Analysis}
\label{chap:empirical-results}

% ====== PREAMBLE ADDITIONS (einmalig in die Präambel) ======
% \usepackage{graphicx}
% \usepackage{float}
% \usepackage{booktabs}
% \usepackage{adjustbox}
% \usepackage{csvsimple}
% % Optionale feine Tabelle:
% % \usepackage{pgfplotstable}

This chapter reports the empirical findings. We begin with per‑run overviews and then synthesize trends across runs. Each overview figure contains three panels that are read left to right: total end‑to‑end time, total proof size, and the time‑efficiency ratio Standard/Nova (values greater than one favor the recursive pipeline). Captions state the resource limits and the configured step size so that runs remain comparable.

\section{Results}

% ---- ONE MACRO PER RUN: custom title + custom limits + observation text + CSV table ----
\newcommand{\ResultRunCustom}[4]{%
  \subsection*{#2}
  \begin{figure}[H]
    \centering
    \includegraphics[width=\linewidth]{ergebnis/md_warm_nr#1/real_crossover_overview.png}
    \caption{#2, limits: \texttt{#3}.}
    \label{fig:run-#1}
  \end{figure}

  #4

  % ---- CSV-Tabelle mit den Messwerten dieses Runs ----
  \begin{table}[H]
    \centering
    % Kurzfassung für List of Tables | Volltext unter der Tabelle
    \caption[Results — Run \##1]{Results — Run \##1 (limits: \texttt{#3})}
    \label{tab:run-#1-summary}
    \begin{adjustbox}{max width=\linewidth}
      \csvautotabular[
        separator=comma,
        head to column names,
        respect all
      ]{data/visualizations/md_warm_nr#1/crossover_summary.csv}
    \end{adjustbox}
  \end{table}

  \vspace{0.75em}
}

% ===== RUNS (Limits je Run hier eintragen) =====
\ResultRunCustom{1}{Run \#1}{--cpus=0.5, --memory=1g, batch size 3}{
\textbf{Observation.} Standard is faster across $N\in[100,1000]$ and the efficiency ratio remains below one. The recursive proof size remains essentially constant while the standard proof size grows roughly linearly. The size crossover appears near eighty two readings. No time crossover is present in this configuration.
}

\ResultRunCustom{2}{Run \#2}{--cpus=1, --memory=2g, batch size 3}{
\textbf{Observation.} Same qualitative pattern up to $N=500$: Standard wins in time; Nova’s proof size is essentially constant ($\approx70$\,KB), Standard grows linearly. Size crossover $\approx 82$; no time crossover.
}

\ResultRunCustom{3}{Run \#3}{--cpus=1, --memory=4, batch size 3}{
\textbf{Observation.} Matches runs \#1 and \#2. The time ratio remains below one throughout and the recursive pipeline is slower on the edge. The size crossover appears around eighty two readings.
}

\ResultRunCustom{4}{Run \#4}{--cpus=1, --memory=8, batch size 3}{
\textbf{Observation.} Standard remains faster end to end and the recursive pipeline produces a single stable proof. The size crossover is close to eighty two readings and no time crossover occurs up to five hundred readings.
}

\ResultRunCustom{5}{Run \#5}{--cpus=2, --memory=2, batch size 3}{
\textbf{Observation.} Time ratio rises with $N$ (Standard approaches Nova) but remains $<1$ for $N\le 500$. Proof-size behavior unchanged (Nova $\approx70$\,KB vs.\ linear Standard). Size crossover near $82$.
}

\ResultRunCustom{6}{Run \#6}{--cpus=4, --memory=2, batch size 3}{
\textbf{Observation.} A time crossover is present and the efficiency crosses one at roughly seven hundred readings in this setting. The recursive pipeline is faster beyond that point. The size crossover remains around eighty two readings.
}

\ResultRunCustom{7}{Run \#7}{--cpus=4, --memory=2g, batch size 10}{
\textbf{Observation.} The time crossover occurs at four hundred readings where the recursive path becomes faster and the advantage increases with input size. Nova’s proof size remains near sixty nine kilobytes while the standard proof grows linearly.
}

\ResultRunCustom{8}{Run \#8}{--cpus=4, --memory=2g, batch size 20}{
\textbf{Observation.} The time crossover appears already at three hundred readings and persists beyond. Larger step size amortizes fixed costs at the price of higher peak memory per step.
}

\ResultRunCustom{9}{Run \#9}{--cpus=0.5, --memory=1g, batch size 20}{
\textbf{Observation.} Under tighter CPU and memory limits the time crossover moves to five hundred readings. The general shape remains consistent and the size behavior is unchanged.
}

\ResultRunCustom{10}{Run \#10}{--cpus=0.5, --memory=1g, batch size 50}{
\textbf{Observation.} With very large steps the time crossover appears at four hundred readings despite tighter CPU and memory. Peak memory per step rises and the recursive prover benefits from fewer folds.
}

\section{Cross-run synthesis}
Across runs \#1–\#10 three stable findings emerge:
\begin{enumerate}
  \item \textbf{Proof size.} Nova’s compressed proof remains essentially constant ($\approx70$\,KB) as $N$ grows, whereas Standard increases roughly linearly at $\sim0.85$\,KB per reading. Hence a size crossover appears around \[
\frac{70\,\text{KB}}{0.85\,\text{KB}} \approx 82~\text{Readings}\] in all runs.

  \item \textbf{Total time.} On constrained edge hardware, Standard is consistently faster for small to medium $N$. With larger $N$ or more favorable limits, the gap narrows. With a step size of ten the time efficiency crosses one at $N=400$ and Nova remains faster beyond that point.
  \item \textbf{Operational trade-off.} Nova minimizes verifier load (one constant-size proof) and can become time-competitive at larger $N$, but incurs higher proving overhead at small $N$. Standard minimizes edge proving time for small/mid batches but yields a linearly growing set of proofs.
\end{enumerate}

\section{Batch size sensitivity}
The recursive pipeline depends on the chosen step size. With a step size of ten the time crossover appears at four hundred readings. Nova is faster from that point and the advantage grows with the input size. Smaller step sizes produce more steps and push the time crossover to the right since the prover spends more work on repeated step transitions. Larger step sizes reduce the number of steps and move the crossover to the left once the prover amortizes the fixed costs of recursion and compression. The choice is a trade off between memory headroom and end to end time. The verifier cost remains constant since a single proof is checked in all cases. The results show that tuning the step size is an effective control to reach a desired operating point on a given device profile.

\section{Summary for the smart-home edge}
For a resource-limited hub aggregating frequent, low-payload readings, the better choice depends on the bottleneck: when edge CPU/RAM and latency dominate, Standard is preferable at small $N$. When consumer bandwidth/verification dominates or when batches become large Nova’s single succinct proof is advantageous and can even overtake Standard in total time once $N$ is sufficiently high (run~\#7).

%%%%%%%%%%%%%%%%%%%%%%%%%%%%%%%%%%%%%%%%%%%%%%%%%%%%%%%%%%%%%%%%%%%%%%%%%%%%%%%%%%%%%%%%%%%%%%%%%%%%
%%%%%%%%%%%%%%%%%%%%%%%%%%%%%%%%%%%%%%%%%%%%%%%%%%%%%%%%%%%%%%%%%%%%%%%%%%%%%%%%%%%%%%%%%%%%%%%%%%%%
%%%%%%%%%%%%%%%%%%%%%%%%%%%%%%%%%%%%%%%%%%%%%%%%%%%%%%%%%%%%%%%%%%%%%%%%%%%%%%%%%%%%%%%%%%%%%%%%%%%%

\chapter{Discussion}
\label{chap:discussion}

\section{Privacy implications of recursion}
Recursive proving reduces the number of proofs a consumer must check to a single constant sized claim that covers many steps. In privacy terms this limits the exposure surface because consumers receive one proof per batch rather than many separate artifacts. The transformation logic and public parameters remain identical to the standard pipeline. The confidentiality guarantee is the same since both pipelines are zero knowledge with respect to the private readings. The practical difference concerns interfaces and logging. A single recursive proof invites simpler handling and avoids per reading side channels in application logs or transport that might correlate with private structure. At the same time recursion changes the amortization of costs at the edge. If batches remain small or if latency is strict then the standard pipeline can be the preferable choice. The privacy benefit of recursion is therefore indirect. It simplifies verification and system interfaces at the consumer, which can reduce accidental information flows in practice, while preserving the same formal guarantees.

\section{Alternative Privacy-Enhancing Technologies}

While our evaluation focuses on zk-SNARKs for verifiable IoT aggregation, several alternative privacy-enhancing technologies (PETs) offer different trade-offs in similar contexts. Understanding these alternatives helps position our contribution within the broader landscape of privacy-preserving IoT systems.

\subsection{Complementary PETs and Their Trade-offs}

\textbf{Differential Privacy} protects individuals by adding calibrated noise to aggregated results, but sacrifices utility and does not guarantee computational integrity \cite{dworkCalibratingNoiseSensitivity2006,dworkAlgorithmicFoundationsDifferential2013}. While effective for statistical privacy, differential privacy cannot provide the exact verification guarantees required for financial or regulatory compliance scenarios.

\textbf{Secure Multi-Party Computation (MPC)} allows computation over distributed inputs without a trusted third party, yet typically incurs high latency and communication overheads that are challenging for constrained IoT devices \cite{yaoHowGenerateExchange1986,goldreichHowPlayANY1987,damgardMultipartyComputationSomewhat2012}. The continuous communication requirements make MPC less suitable for the intermittent connectivity patterns common in IoT deployments.

\textbf{Trusted Execution Environments (TEEs)} provide hardware-backed isolation for secure computation, but introduce additional trust assumptions and are vulnerable to side-channel attacks. Furthermore, TEE availability varies significantly across IoT device classes, limiting deployment flexibility.

\textbf{ZK-STARKs} offer transparency and post-quantum security, but generally require larger proof sizes and higher prover costs compared to SNARK systems, making them less suitable for bandwidth-constrained IoT environments.

\subsection{Why zk-SNARKs for IoT Aggregation}

Given our goal of verifiable aggregation under device and bandwidth constraints, zk-SNARKs provide the optimal balance of proof succinctness, verification efficiency, and privacy guarantees. Our empirical evaluation demonstrates that recursive zk-SNARKs (Nova) can achieve constant-size verification while maintaining computational integrity, making them particularly suitable for resource-constrained IoT environments.


%%%%%%%%%%%%%%%%%%%%%%%%%%%%%%%%%%%%%%%%%%%%%%%%%%%%%%%%%%%%%%%%%%%%%%%%%%%%%%%%%%%%%%%%%%%%%%%%%%%%
%%%%%%%%%%%%%%%%%%%%%%%%%%%%%%%%%%%%%%%%%%%%%%%%%%%%%%%%%%%%%%%%%%%%%%%%%%%%%%%%%%%%%%%%%%%%%%%%%%%%
%%%%%%%%%%%%%%%%%%%%%%%%%%%%%%%%%%%%%%%%%%%%%%%%%%%%%%%%%%%%%%%%%%%%%%%%%%%%%%%%%%%%%%%%%%%%%%%%%%%%

\chapter{Conclusion \& Future Work}
\label{chap:conclusion}
This thesis examined whether recursive zk-SNARKs are practically beneficial for verifiable IoT aggregation on constrained edge hardware. We implemented two pipelines that share the same transformation semantics and trust assumptions: a standard Groth16 path and a recursive Nova path that optionally compresses the accumulated recursion state to a single standard-compatible proof. Both pipelines provide the same guarantees. The standard path scales time and artifact size with the number of inputs, while the Nova path concentrates effort in proving and leaves one constant‑size verification for the consumer.

Our measurements on a dockerized edge environment with explicit CPU and memory limits show a consistent pattern. Proof size is essentially constant for the recursive pipeline while it grows roughly linearly for the standard pipeline, which yields a size crossover around 82 readings in all configurations we tested. Total time on the edge favors the standard pipeline for small to medium input sizes. As batches grow and with a step size of ten, the time efficiency crosses one at four hundred readings and the recursive pipeline becomes faster beyond that point. Larger step sizes shift the crossover to smaller input sizes by amortizing fixed folding costs, at the expense of higher peak memory and fewer opportunities to stream or preempt. Tighter CPU and memory limits shift the crossover to the right as expected, but do not change the qualitative picture. These results confirm that recursion can deliver constant verification cost and competitive end-to-end latency once inputs are sufficiently large or when consumer-side constraints dominate.

The architectural implications are straightforward. When low latency for small batches is critical or hardware is very weak, the standard pipeline remains the pragmatic choice. When verification and bandwidth at the consumer are the bottleneck, or when aggregation naturally yields larger batches, the recursive pipeline provides operational simplicity through a single succinct proof and can become time competitive or superior. The choice of step size is a tunable control: increasing it reduces per-batch overhead and advances the time crossover, but raises peak prover memory and increases the cost of aborts and padding. In our environment step size ten struck a robust balance.

From a privacy perspective both pipelines provide the same formal guarantees since they differ only in proof composition. In practice the recursive pipeline simplifies interfaces by exposing a single proof per batch, which reduces ancillary metadata and potential side channels in application logs or transport.

This work has limitations. The evaluation uses synthetic yet structured sensor traces, a single host platform with docker-imposed CPU and memory limits, and a specific tooling stack and curve (ZoKrates 0.8.8 on Pallas). Caching and I/O can bias timings and we mitigate this by keeping the environment stable and by averaging multiple replicates. These factors bound external validity; however, the qualitative findings about size constancy, crossover shifts with step size, and verifier cost concentration are robust across our runs.

Practitioners can apply three concrete guidelines. Use the standard pipeline for small batches or strict tail latency on weak edge devices. Use the recursive pipeline for batches beyond the measured crossover or when consumer bandwidth and verification time dominate, and prefer the compressed variant for portability. Tune step size to meet memory headroom while seeking earlier time crossovers; in our setting step size ten performed well, step size twenty advanced the crossover further at the cost of higher peak memory, and step size five delayed it.

Future work should bind input sets cryptographically and externally via Merkle commitments to support selective disclosure and end-to-end set attestation. A fully external verifier service for consumers would decouple verification from the edge environment. Exploring alternative folding schemes and transparent proof systems may further reduce proving latency without sacrificing succinct verification. Finally, extending the evaluation to heterogeneous devices and real sensor deployments would strengthen the operational guidance and refine step size policies under diverse workloads.

\printbibliography

\end{document}


